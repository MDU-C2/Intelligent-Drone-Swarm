\section{Related Work}
% Relevant work: distributed consensus algorithms (e.g., Paxos, Raft, Byzantine fault tolerance), UAV swarm resilience and cooperative autonomy, safety-critical systems, safety-critical dependability approaches in distributed systems.

\begin{table}[H]
    \centering
    \begin{tabular}{|B{0.7cm}|C{5cm}|C{4cm}|C{4cm}|}
        \hline
        \textbf{Year} & \textbf{Title} & \textbf{Main Focus} & \textbf{Relevance to \acrshort{irds}} \\
        \hline
        2025 & Design of a Fail-Operational Swarm of Drones for Search and Rescue Missions \cite{LuizNotPublished} & High-level blueprint for a conceptual architecture for a fail-operational UAV swarm. & Degraded health message. \\
        \hline
        2023 & Cooperative Search and Rescue with Drone Swarm \cite{Luiz2023} & Distributed coordination and search resilience in UAV SAR missions. & Defines the application domain and justifies decentralised swarm cooperation for dependability. \\
        \hline
        2021 & Autonomous and Collective Intelligence for UAV Swarm in Target Search Scenario \cite{Luiz2021} & Conceptual architecture for autonomous \& collective intelligence in target search. & Serves as a theoretical base. \\
        \hline
        2025 & Development of Adaptive Drone Swarm Networks \cite{ChenYang-Yi2025DoAD} & Adaptive network layer (AeroSyn) & Suggests implementation model for health-state messaging \\ %Chen et al. (2025)
        \hline
        2025 & Dynamic reconnaissance operations with UAV swarms: adapting to environmental changes \cite{StodolaPetr2025Drow} & Dynamic replanning under failure & Provides comparative benchmark and validation model \\ %Stodola et al. (2025)
        \hline
        2025 & Energy Efficient Scheduling for Position Reconfiguration of Swarm Drones \cite{LiuHan2025EESf} & Energy-efficient reconfiguration & Supports energy-aware task redistribution \\ %Liu et al. (2025)
        \hline
        2023 & A Novel Distributed Situation Awareness Consensus Approach for UAV Swarm Systems \cite{HaiXingshuo2023ANDS} & Distributed consensus & Algorithmic basis for swarm agreement \\ %Hai et al. (2023)
        \hline
        2022 & A Review of Consensus-based Multi-agent UAV Implementations \cite{LizzioFaustoFrancesco2022ARoC} & Consensus-based UAV control & Practical implementation insights \\ %Lizzio et al. (2022)
        \hline
        2021 & Review of Dynamic Task Allocation Methods for UAV Swarms Oriented to Ground Targets \cite{QiangPeng2021RoDT} & Dynamic task allocation & Framework for reallocation strategy \\ %Peng et al. (2021)
        \hline
        2019 & UAV swarm communication and control architectures: a review \cite{CampionMitch2019Usca} & Swarm communication architectures & Justifies decentralised, \acrshort{fanet}-based design \\ %Campion et al. (2018)
        \hline
        %1 & 2 & 3 & 4 \\
        %\hline
    \end{tabular}
    \caption{Caption}
    \label{tab:placeholder}
\end{table}

\subsection{Cooperative Search and Rescue with Drone Swarm}
This paper presents a cooperative search-and-rescue (SAR) strategy using UAV swarms.
It explores collective intelligence, distributed coordination, and fault-tolerant cooperation among \acrshortpl{uav} in dynamic SAR missions. The swarm shares situational data to maintain coverage and efficiency while adapting to \acrshort{uav} losses or degraded conditions.

\subsection{Autonomous and Collective Intelligence for UAV Swarm in Target Search Scenario}
This paper investigates collective intelligence architectures for UAV swarms performing target-search missions.
It introduces autonomous decision layers combining individual UAV autonomy (sensing, navigation) with collective coordination for search coverage.

\subsection{Development of Adaptive Drone Swarm Networks}
This paper introduces AeroSyn, a hybrid network architecture that combines cellular and \gls{adhoc} links. It enables \acrshortpl{uav}s to switch between connected and leader-follower communication modes to maintain stable swarm coordination.

\subsection{Dynamic reconnaissance operations with UAV swarms: adapting to environmental changes}
This paper develops a dynamic mission-replanning framework for UAV swarm reconnaissance using \acrfull{aco}. The model handles two events that force the system to adapt:
\begin{enumerate}
    \item Swarm composition changes (e.g., \acrshort{uav} failures/additions).
    \item Mission or environment changes (e.g., new area of responsibility).
\end{enumerate}

\subsection{Energy Efficient Scheduling for Position Reconfiguration of Swarm Drones}
This study proposes an energy-balancing reconfiguration scheme for \acrshort{uav} swarms operating in urban wind fields. The model determines when and how \acrshortpl{uav} should exchange positions to equalise energy use and extend swarm flight time.

\subsection{A Novel Distributed Situation Awareness Consensus Approach for UAV Swarm Systems}
This paper proposes a distributed situation awareness consensus framework that allows UAVs to share perception and align decision-making without a central controller. It introduces a dual-loop feedback mechanism to maintain robust cognitive agreement.

\subsection{A Review of Consensus-based Multi-agent UAV Implementations}
This review analyses consensus-based distributed control for \acrshort{uav} swarms. The review highlights practical issues in formation flight and target tracking, focusing on how communication delay and sensing accuracy affect practical use.

\subsection{Review of Dynamic Task Allocation Methods for UAV Swarms Oriented to Ground Targets}
This review surveys dynamic task allocation methods for \acrshort{uav} swarms facing changing ground-target situations. The review classifies dynamic task allocation models into global and local approaches and analyses algorithms including market-based, intelligent optimisation, and clustering strategies. Each method’s advantages and practical performance trade-offs are discussed. The review identifies ongoing challenges such as communication limits and heterogeneous \acrshort{uav} swarms, and suggests future integration of AI and distributed optimisation for adaptive task allocation.

\subsection{UAV swarm communication and control architectures: a review}
This paper reviews UAV swarm communication and control architectures by comparing centralised (ground-based) and \gls{adhoc} (\acrshort{fanet}) models. The paper discusses autonomy levels, swarm coordination algorithms, and the potential of 5G networks to improve UAV-to-UAV communication. The paper highlights the need for hybrid architectures that combine distributed decision-making with reliable infrastructure support to overcome range, latency, and scalability issues.