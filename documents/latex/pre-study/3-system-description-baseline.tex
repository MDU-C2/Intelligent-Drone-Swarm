\section{Baseline System Architecture \& IRDS Concept}

The baseline system consists of a UAV swarm executing a SAR mission, a human Search Manager, and supporting infrastructure such as the simulation environment and communication links. At this stage, the Swarm Coordination Module (SCM) is conceptual rather than fully specified, but it can already be decomposed into a set of logical sub-functions that reflect the core challenges identified in the background and related-work analysis.

In the baseline architecture (\cref{fig:app-system-baseline}), each UAV is equipped with a Fault Management Unit (FMU) that monitors onboard hardware and reports a health status. These health reports, together with task and sector information, are processed by the SCM, which spans the swarm and is realised cooperatively by all agents rather than as a single central component. Conceptually, the SCM is expected to include at least the following subcomponents:
\begin{itemize}
    \item State \& Health Monitor – collects health information from FMUs and mission state data (sector ownership, role assignments, etc.) and exposes an abstract, swarm-level view of agent status.
    \item Consensus Module – implements a distributed mechanism for reaching agreement on critical events (e.g. agent degraded, task handed over) under intermittent connectivity and possible faulty reports.
    \item Task \& Sector Allocation Module – manages the assignment of search sectors and other tasks to agents, including reallocation when health changes, with the goal of preserving coverage and workload balance.
    \item Role Allocation / Behaviour Module – determines which high-level role each agent should perform (e.g. search, relay, return-to-base, standby) based on its health, assigned sectors, and mission phase.
    \item Communication \& Networking Interface – abstracts the underlying flying ad-hoc network, providing message dissemination and routing services required by the consensus and allocation logic.
\end{itemize}


The current baseline diagram will be refined in later phases, but already at the pre-study stage it is important to recognise that the IRDS ''box'' inside the SCM is not a monolithic entity. Instead, it is expected to be realised as a cooperation of these subcomponents, each addressing a specific technical challenge. The subsequent System Design Description will formalise these modules, define their interfaces, and allocate requirements to them.

\begin{figure}[H]
    \centering
    \includegraphics[width=0.8\linewidth]{system-architecture-baseline.png}
    \caption{Baseline system architecture.}
    \label{fig:system-baseline}
\end{figure}

A larger version of figure \ref{fig:system-baseline} can be found in the Appendix (figure \ref{fig:app-system-baseline}).