\section{Dependability Perspective}
%The \acrshort{irds} project is focused on dependability, and the project's success depends on the swarm’s ability to complete \acrshort{sar} missions in uncertain and dynamic environments, even when individual \acrshortpl{uav} degrade. The following dependability properties are central to the project:
%\begin{itemize}
%    \item Reliability
%    \item Availability
%    \item Safety
%\end{itemize}

%Those attributes shall ensure that the replanning protocol design does not focus only on functionality, but also incorporates dependability as an objective. Each of the dependability properties mentioned are achieved by the project’s management plans (Safety, Quality, Requirements, V\&V, and Configuration) to guarantee consistency and traceability during the development process.
For the IRDS project, dependability is not achieved by management plans alone; it must be built into the technical design of the replanning protocol and the swarm architecture. In particular, the system must tolerate individual UAV failures, detect and contain faults, and maintain coordinated behaviour even under degraded conditions. Management plans (requirements, safety, V\&V, quality, configuration) support these goals by structuring the process, but the underlying capability comes from architectural choices such as redundancy, robust consensus mechanisms, and fault-aware task allocation.

This section outlines how reliability, availability, and safety are interpreted for IRDS and specifies initial quantitative targets that will later be refined and validated in simulation.

\subsection{Reliability}
%The \acrshort{uav} swarm must continue to perform its \acrshort{sar} mission despite the loss or degradation of one or more \acrshortpl{uav}. Reliability will be achieved by implementing distributed consensus and task reallocation mechanisms that prevent mission failure when \glspl{agent} are compromised. The replanning protocol must tolerate both single-\acrshort{uav} faults and swarm-level faults. This includes graceful performance degradation when a \acrshort{uav} is partially functional and collective resilience when UAVs are lost. Fault injection in simulation will be used to validate the system’s robustness against such conditions.
In the IRDS context, reliability refers to the probability that the swarm completes its SAR mission objectives despite individual agent degradations or failures. Technically, this will be pursued by:
\begin{itemize}
    \item Designing the protocol so that critical mission information is replicated across agents, avoiding single points of failure.
    \item Using a distributed consensus mechanism that tolerates communication loss and faulty health reports within defined bounds.
    \item Ensuring that task and sector ownership can be reassigned when an agent fails, without leaving sectors permanently unsearched.
\end{itemize}
%As an initial quantitative target, the project aims for a minimum mission success rate of 95% under the assumed fault model in simulation, where mission success is defined as completing the planned search coverage within acceptable time and safety constraints even when a subset of agents degrade or fail.

\subsection{Availability}
%The \acrshort{uav} swarm should remain operational as long as possible, ensuring that \acrshort{sar} coverage is maintained unless the number of failed drones reaches a critical threshold. The ability to retask degraded \acrshortpl{uav} as communication relays or in less demanding roles supports availability by ensuring continued contribution of partially functioning \glspl{agent}.
Availability captures the ability of the swarm to remain operational over time, even when individual agents must be removed, return to base, or switch to secondary roles. The IRDS protocol contributes to availability by:
\begin{itemize}
    \item Allowing agents to continue participating in the swarm in reduced-capability roles (e.g. communication relay) rather than dropping out entirely.
    \item Supporting on-line task redistribution so that coverage is maintained when agents become unavailable.
    \item Minimising the time between fault detection and completion of task handover.
\end{itemize}

To make availability measurable, the following could be defined: Targets such as a maximum allowable downtime per drone (time between a failure or degradation event and the swarm regaining a complete and consistent task allocation) and a limit on the fraction of mission time during which required coverage is not maintained due to reconfiguration.

\subsection{Safety}
%Replanning must not introduce unsafe behaviours, such as mid-air collisions or conflicting tasks. Safety shall be considered by following established standards (ARP4754A, ARP4761A, and SORA), and by defining safety requirements that prevent hazardous states.
Safety is concerned with preventing the replanning protocol from introducing hazardous behaviours, such as collisions, loss of separation, or uncontrolled flight due to inconsistent commands. From a technical point of view, this means that:
\begin{itemize}
    \item Fault detection and consensus must be designed so that no single faulty agent can drive the swarm into an unsafe state within the assumed fault model.
    \item Task reallocation must respect spatial and temporal safety constraints, such as minimum separation distances and maximum allowed overlap in sectors.
    \item The protocol should provide guaranteed safe task reassignment, meaning that agents only act on commands that have passed defined consistency and plausibility checks.
\end{itemize}


These safety aims will be supported and refined through the Safety Management Plan and preliminary safety analyses, but their realisation ultimately depends on protocol properties that can be analysed and tested using model-based verification and fault-injection simulation.

%Reliability
%Availability
%Maintainability %Maintainability - Går bort för att vi inte har ett fysiskt system.
%Safety
%(SECURITY)
%Confidentiality in information security.
%Integrity in information security.
%Availability in information security.