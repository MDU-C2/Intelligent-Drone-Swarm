\section{Roles \& Responsibilities}
The IRDS project is organised according to a role-based structure. Each role has clearly defined responsibilities and authority to avoid overlaps and ambiguity.
\begin{itemize}
    \item Chief Engineer (CE)
    \begin{itemize}
        \item[$\circ$] Provides overall technical leadership and ensures that all work aligns with the system concept, dependability objectives, and project scope.
        \item[$\circ$] Has final approval authority on system-level design choices, architectural decisions, and trade-offs between competing dependability attributes.
        \item[$\circ$] Integrates results from all other roles.% and acts as the main point of contact with the project owner and examiners.
    \end{itemize}
    \item Requirements Manager (RM)
    \begin{itemize}
        \item[$\circ$] Elicits, documents, and maintains requirements, with a focus on clarity, feasibility, and verifiability.
        \item[$\circ$] Ensures traceability between stakeholder needs, system functions, and dependability objectives.
        \item[$\circ$] Has authority over the requirements baseline: May approve or reject changes to requirements and request clarifications from other roles.
        \item[$\circ$] Collaborates with the SM and VVM, but final decisions on requirements content rest with the RM, subject to CE arbitration in case of conflict.
    \end{itemize}
    \item Safety Manager (SM)
    \begin{itemize}
        \item[$\circ$] Identifies hazards and defines safety goals and safety requirements.
        \item[$\circ$] Ensures that safety considerations are incorporated into the design of the replanning protocol and the system architecture.
        \item[$\circ$] Has authority over safety-related decisions (e.g. acceptance of safety mitigations, safety goals), subject to alignment with project scope and design as agreed with the CE.
    \end{itemize}
    \item Validation \& Verification Manager (VVM)
    \begin{itemize}
        \item[$\circ$] Defines the V\&V strategy, including test cases, fault-injection scenarios, and acceptance criteria.
        \item[$\circ$] Ensures that all requirements are verifiable and that V\&V results are traceable to requirements and design artefacts.
        \item[$\circ$] Has authority to accept or reject deliverables based on verification and validation results and to request corrective actions.
        \item[$\circ$] Validates that the implemented protocol meets stakeholder needs from a V\&V perspective, while recognising that final design authority remains with the CE.
    \end{itemize}
    \item Quality \& Configuration Manager (QCM)
    \begin{itemize}
        \item[$\circ$] Maintains consistency and traceability across all project artefacts through configuration management.
        \item[$\circ$] Manages version control, document reviews, and configuration baselines.
        \item[$\circ$] Has authority to approve changes to controlled documents and baselines and to reject outputs that do not meet agreed quality criteria.
    \end{itemize}
\end{itemize}
To avoid ambiguity, conflicts between roles are resolved as follows:
\begin{itemize}
    \item The CE is the final decision-maker for system-level design and technical trade-offs.
    \item The RM has authority over the content and structure of requirements, while the VVM has authority over whether those requirements are adequately verified and validated.
    \item The SM has authority over safety goals and safety requirements; If a safety concern conflicts with other project objectives, it is escalated to the CE for resolution.
    \item The QCM may block the release of artefacts that do not meet quality or configuration-control criteria, but cannot override technical design decisions; Such disagreements are also escalated to the CE.
\end{itemize}

This division of responsibilities is intended to ensure clear ownership of decisions while preserving efficient coordination and avoiding overlaps that could slow down the project.