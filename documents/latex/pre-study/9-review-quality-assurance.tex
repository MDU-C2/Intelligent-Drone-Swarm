\section{Review \& Quality Assurance}
\color{red}
\begin{itemize}
    \item Each deliverable is peer-reviewed by another group member.
    \item Changes and comments documented.
    \item Artefacts tracked in GitHub/Teams.
\end{itemize}

\begin{itemize}
    \item Each deliverable will undergo peer review by at least one other team member.
    \item Comments and changes will be documented in GitHub.
    \item Artefacts are tracked in GitHub/SharePoint depending on type.
\end{itemize}
Detailed review and quality processes are defined in QM-01 and summarised in the Project Plan (Sections 6.4–6.5).
\color{black}

All project deliverables are subject to structured review and approval, as defined in the Quality Management Plan (QM-01). Quality assurance is embedded throughout the project lifecycle to ensure that documents, code, and analyses are consistent, traceable, and aligned with recognised standards.

Review Procedures

Peer Review: Each deliverable undergoes peer review by at least one team member from a different role, ensuring cross-disciplinary scrutiny.

Formal Review: Drafts are reviewed against checklists derived from ISO 9001, IEEE 730, and project-specific templates, focusing on correctness, completeness, and compliance with standards.

Approval: The Chief Engineer signs off final versions once reviewed and approved by the relevant managers (RM, SM, QCM, VVM).

Quality Assurance Activities

Traceability checks: Ensuring links between requirements, safety goals, V\&V activities, and outcomes are maintained.

Configuration audits: The QCM verifies that documents, simulations, and results are properly version-controlled in Git and stored in shared repositories.

Corrective actions: Identified issues are logged, discussed, and resolved, with updates tracked in version history.

Continuous improvement: Lessons learned from reviews are incorporated into subsequent deliverables to improve efficiency and quality.

Integration with Management Plans

The Requirements Manager ensures that requirements remain consistent and verifiable.

The Safety Manager confirms that hazard analyses and safety objectives are properly reflected in the design.

The V\&V Manager ensures that validation metrics are correctly applied in testing.

The QCM coordinates the process, ensuring consistency and proper documentation.

By combining peer review, formal review, and structured approval with configuration management and corrective action tracking, the project guarantees that its outputs are not only technically sound but also meet the quality expectations of the course and the dependability standards adopted.
