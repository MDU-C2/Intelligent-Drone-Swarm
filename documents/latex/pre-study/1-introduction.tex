\section{Introduction}
\begin{itemize}
    \item \textbf{Project title:} \acrfull{irds}.
    \begin{itemize}
        \item[] \textit{Orignal title: Intelligent Replanning Protocol for a Fail-Operational Drone Swarm.}
    \end{itemize}
    \item \textbf{Project owner:} Luiz Giacomossi
\end{itemize}

\vspace{0.5cm}

\noindent \acrfull{sar} missions increasingly rely on autonomous multi-agent systems that can cover large areas quickly while maintaining an acceptable level of safety and reliability. \acrfull{uav} swarms offer clear advantages over single-UAV deployments in terms of robustness and scalability, but their effectiveness is still limited by the health of individual agents and by the difficulty of coordinating many autonomous units in uncertain environments.

The \acrfull{irds} project addresses this challenge by investigating how a UAV swarm can remain fail-operational when one or more agents experience degraded health or fail entirely. Rather than allowing failed agents to silently drop out, the swarm should collectively detect degradations, redistribute tasks, and, where possible, assign secondary roles to compromised agents so that overall mission performance is preserved. Achieving this requires more than ad-hoc behaviours: it demands a structured protocol that combines distributed consensus, dynamic task allocation, and dependable communication.

This pre-study lays the analytical foundation for such a protocol. It frames the problem in the context of dependability engineering, reviews related work on UAV swarms, consensus mechanisms, and task allocation, outlines a baseline system architecture, and defines initial dependability objectives and constraints. The results of this pre-study guide later phases of the project, where the conceptual protocol will be specified, partly implemented and validated through fault-injection in a simulation environment.

\subsection{Background}

The \acrfull{irds} project will address the challenge of maintaining mission continuity in \acrfull{sar} operations when individual \acrfullpl{uav} suffer degraded health or fail entirely. Previous work by \cite{LuizNotPublished} established a high-level blueprint for a safety-driven conceptual architecture for a fail-operational \acrshort{uav} swarm intended for \acrshort{sar} missions. The architecture contained a \acrfull{fmu} to detect hardware component failures and broadcast a degraded health message across the swarm. The next step is to transform this information into a protocol for collective replanning.

In dynamic and uncertain environments, conventional single-\acrshort{uav} fault tolerance is insufficient. Instead, swarms must demonstrate fail-operational behaviour by continuing their mission despite the loss or degradation of one or more \glspl{agent}. This requires distributed consensus, adaptive task allocation, and resilient communication. The project shall build on current research on swarm resilience, distributed consensus, and communication architectures while focusing on dependable design and validation.

The \acrshort{irds} project builds on this foundation by transforming that degraded health message into swarm-level replanning logic to enable the swarm to continue its \acrshort{sar} effectively even when one or more \acrshortpl{uav} are degraded.

To achieve this, the following challenges must be addressed:
\begin{enumerate}
    \item Distributed Consensus
    \item Dynamic Task Allocation
    \item Protocol Design
    \item Validation
\end{enumerate}

Distributed consensus is particularly challenging in the IRDS context because the swarm operates over wireless, potentially intermittent links, without a permanently available central controller or global clock. Agents may temporarily lose connectivity, rejoin the swarm, or report inconsistent health information, yet the swarm must still converge on a coherent view of mission state and required actions. Consensus algorithms therefore need to tolerate message delays, packet loss, and faulty or misleading health reports while remaining lightweight enough for resource-constrained UAV hardware.

Dynamic task allocation is likewise non-trivial. In SAR missions, tasks correspond to coverage of specific sectors, and reassigning them after failures must balance multiple objectives: Maintaining coverage of high-priority areas, respecting the limited energy and flight envelopes of individual UAVs, and avoiding unsafe overlaps or gaps. The allocation logic must react to health changes and failures in real time, without central optimisation, and must do so in a way that is consistent with swarm-level safety and reliability goals.

The contribution of the project team will be to create a decentralised replanning protocol that allows a degraded UAV to hand over its tasks, assume a secondary role, and allow the swarm to reach consensus on what to do – partly validated in simulation with fault injection.

%The contribution of the project team will be to create a decentralised replanning protocol that allows a degraded \acrshort{uav} to hand over its tasks, assume a secondary role, and allow the swarm to reach consensus on what to do -- all validated in simulation with fault injection.

\subsection{Purpose}

The purpose of this pre-study is to establish the foundations for a decentralised replanning protocol that ensures \acrshort{sar} missions can continue with maximum effectiveness despite \acrshort{uav} degradation. This includes:
\begin{itemize}
    \item Review prior work on consensus, task allocation, and communication in \acrshort{uav} swarms.
    \item Define the conceptual system description and dependability objectives.
    \item Identify relevant standards and regulations to guide the project.
    \item Outline planned activities, roles, and deliverables for the project.
\end{itemize}

\subsection{Scope}
The scope of this project is limited to the design, simulation (using gym-pybullets-drone \cite{pybullets}), and validation and verification of the replanning protocol. The project will establish what the \acrshort{uav} swarm will do, not how to do it. Hardware implementation and certification are excluded. The replanning protocol will be validated using a modified version of the simulation software, using fault injection, to test the performance of the replanning protocol. The outcome will be a validated protocol design, documented processes, and assurance artefacts in accordance with the FLA402 course requirements and course guide \cite{study-guide}.


%Struktur, introduktion (backgroound, purpose, scope), System Descirption, Related Work, Proposed Solution