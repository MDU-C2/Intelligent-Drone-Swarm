\section{Standards \& Regulations}
The selection of standards (\cref{tab:rel-standards}) for the project ensures that the project is consistent with recognised international standards and guidelines.
\newcolumntype{P}[1]{>{\raggedright\arraybackslash}p{#1}}
\renewcommand{\arraystretch}{1.4}
\begin{table}[H]
    \centering
    \begin{tabular}{|P{4.5cm}|P{4cm}|P{7cm}|}
        \hline
        \textbf{ID} & \textbf{Title} & \textbf{Relevance} \\
        \hline
        IEEE Std 1012\texttrademark-2024 & IEEE Standard for System, Software, and Hardware Verification and Validation \cite{1012_2024} & Supports verification and validation activities across all project phases, ensuring that both system and software meet their specified requirements and intended use. \\
        \hline
        ISO/IEC/IEEE Std 29119-1:\texttrademark-2022 & Software and systems engineering - Software testing - Part 1: General Concepts \cite{29119_1_2022} & Provides the foundational concepts and high-level framework for software testing, helping the project understand what aspects of the system must be tested. \\
        \hline
        ISO/IEC/IEEE Std 29119-4:\texttrademark-2021 &  Software and systems engineering - Software testing - Part 4: Test techniques \cite{29119_4_2021} & Provides test design techniques to ensure systematic and effective test coverage of the system. \\
        \hline
        ISO/IEC/IEEE Std 15288:\texttrademark-2023 & Systems and software engineering - System life cycle processes \cite{15288_2023} & Defines processes for the full system lifecycle, including development, verification, and validation. \\
        \hline
        ISO/IEC/IEEE 29148:2018 & Systems and software engineering - Life cycle processes - Requirements engineering \cite{29148_2018} & Supports the project's requirements engineering activities by providing guidance for creating clear, consistent, and verifiable system and software requirements. \\
        \hline
        ISO 10007:2017 & Quality management - Guidelines for configuration management \cite{10007_2017} & Useful for ensuring proper configuration management throughout the project lifecycle, supporting traceability, version control, and change management. \\
        \hline
        ISO 9001:2015 & Quality management systems — Requirements \cite{9001_2015} & Manages overall project quality by establishing requirements for a quality management system to ensure consistent, reliable processes and deliverables. \\
        \hline
        ISO/IEC 25002:2024 & Systems and software engineering — Systems and software Quality Requirements and Evaluation (SQuaRE) — Quality model overview and usage \cite{25002_2024} & Supports the project's software architecture work by providing guidance on evaluating software product quality, helping ensure the system meets defined quality characteristics. \\
        \hline
        IEEE 730-2014 & IEEE Standard for Software Quality Assurance Processes \cite{730_2014} & Defines the requirements for a Software Quality Assurance Plan (SQAP), supporting structured and traceable software development processes. \\
        \hline
        JAR-DEL-SRM-SORA-MB-2.5 & Specific Operations Risk Assessment (SORA) \cite{jarus_sora} & This methodology ensures that the system complies with the legal and safety requirements for conducting risk-based assessments of unmanned aircraft operations within Europe. \\
        \hline
        ARP4761\texttrademark A & Guidelines for Conducting the Safety Assessment Process on Civil Aircraft, Systems, and Equipment \cite{sae_arp4761a} & Provides guidance on performing safety assessments to identify hazards and define corresponding safety requirements for aerial vehicles. \\
        \hline
    \end{tabular}
    \caption{Selected standards and guidelines.}
    \label{tab:rel-standards}
\end{table}
\renewcommand{\arraystretch}{1.0}
%ARP4754A (SAE) – Guidelines for the development of civil aircraft and systems. Applied here at a conceptual level to structure the system development and safety assessment.

%SORA (JARUS) – Specific Operations Risk Assessment framework for UAV operations, guiding the identification and mitigation of risks specific to drone swarms.

%Other safety-related standards such as DO-178C (software certification for airborne systems) and IEC 61508 (functional safety) are recognised as important references in the broader aerospace and safety engineering context. However, they are not directly applied in this project, since its scope is limited to the design and validation of a simulation-based protocol rather than certifiable airborne software.

