\section{System Architecture}
\label{sec:system-architecture}
This section describes the structural architecture of the IRDS protocol module and its surroundings. It identifies the main components, their internal sub-modules, and the interfaces through which they interact with one another and with external systems such as the SwarmInterface and the UAV flight stack. The section also introduces the communication model and information model used by the protocol. Together, these views provide the static foundation on which the dynamic behaviours in Sections \ref{sec:rumour-mill} to \ref{sec:task-role} are defined.

\subsection{Architecture Context}
At the top level, the IRDS architecture consists of:
\begin{itemize}
    \item A Swarm of 4–255 agents, each running the IRDS protocol module.
    \item A SwarmInterface that relays high-level inputs from the Search Manager and displays swarm status.
    \item An Environment that influences communication and flight.
\end{itemize}
The System Context Diagram (\cref{fig:bdd-context}) shows this context:
\begin{itemize}
    \item The IRDS System of Systems interacts with the environment.
    \item The Search Manager interacts only with the SwarmInterface.
    \item The SwarmInterface exchanges messages with the swarm as a whole.
    \item The swarm is modelled as a composition of agents.
    \item Each agent contains an \acrshort{fmu} and the IRDS protocol module.
\end{itemize}
Therefore, the protocol module is a swarm-level coordination layer that sits on top of the UAVs’ flight and navigation systems.


\subsection{Decentralised Swarm Structure}

\subsubsection{Swarm \& Neighbourhood Composition}
The swarm is decentralised:
\begin{itemize}
    \item There is no central controller or privileged leader.
    \item All agents run the same protocol module.
    \item Each agent maintains its own local view of the swarm (AgentInfo, SectorInfo, neighbourhoods).
    \item Swarm-wide behaviour emerges from repeated local interactions and the Rumour Mill.
\end{itemize}
If an individual agent fails or leaves, the remaining agents:
\begin{itemize}
    \item Changes its membership status in their AgentInfo lists.
    \item Recompute affected neighbourhoods.
    \item Release and reallocate its \glspl{sector} via the Sector Allocation Market.
    \item Continue the mission without requiring any central reconfiguration.
\end{itemize}
Each fully operational agent maintains a neighbourhood of four fully operational agents in total consisting of itself and its three geographically closest agents (where ''closest'' is defined using the centres of each agent’s initial \glspl{sector}), and any partially operational agents.

\vspace{0.2cm}

\subsubsection{Neighbourhood Formation at Mission Start}
\begin{figure}[H]
    \centering
    \includegraphics[width=\linewidth]{SD-24_AD - Initial Neighbourhoods 1.0.png}
    \caption{\textit{(SD-24)} \gls{actdia} showing the process of neighbourhood formation after initial sectors have been assigned.}
    \label{fig:ad-initial-neighbourhoods}
\end{figure}


\subsubsection{Neighbourhood updates after join/re-join}
\begin{figure}[H]
    \centering
    \includegraphics[width=0.8\linewidth]{SD-25_AD - Recalculate Neighbourhoods 1.0.png}
    \caption{\textit{(SD-25)} \gls{actdia} showing the process of neighbourhood updates after agents join or re-join after mission start.}
    \label{fig:ad-neighbourhood-updates}
\end{figure}


\subsubsection{Overlapping Neighbourhoods}
Neighbourhoods overlap, meaning that an agent is part of its own neighbourhood (as the centre agent), and it also appears as a neighbour in the neighbourhoods of other agents.

This overlap creates a connected graph of small local groups throughout the swarm. Origin rumours are broadcast to neighbours, neighbours broadcast Source rumours to their neighbours, which enables the Source rumours to propagate through the swarm; Opinion rumours then trigger the originator to make a decision.

Neighbourhoods do not perform any internal voting. Each neighbour independently:
\begin{itemize}
    \item Receives the Origin,
    \item interprets it using local information,
    \item broadcasts a Source.
\end{itemize}
%\subsubsection{Neighbourhood Operation}
Each fully operational agent forms a neighbourhood with its three geographically closest fully operational peers (based on assigned initial sectors), and possibly absorbs partially operational agents into its neighbourhood. Neighbourhoods overlap, provide multiple independent Source rumour for each Origin rumour, and serve as the initial ''cross-check'' group for validating events.

%\vspace{0.2cm}
\newpage

\subsubsection{Neighbourhood Recalculation on Failure or Leave}
Neighbourhoods must be updated when:
\begin{itemize}
    \item An agent leaves the swarm intentionally (e.g. battery replacement).
    \item An agent fails or is declared missing due to lost pulses.
    \item A new agent joins the swarm.
    \item An agent rejoins the swarm.
\end{itemize}
In these cases, each affected agent that is fully operational recomputes its three closest fully operational neighbours based on the latest available sector or position information. This ensures that neighbourhoods maintain their size (1 + 3) and that cross-checking coverage remains consistent.

Neighbourhoods do not vote as a group; Each neighbour independently creates Source rumours. The requirement to receive three Source rumours before interpeting an event and possibly broadcasting an Opinion gives each neighbourhood a built-in tolerance to some faulty behaviour, provided that no neighbourhood contains too many faulty agents.


\subsection{Agent Architecture}
Each agent is a \acrshort{uav} equipped with standard flight systems (example found in \cref{fig:bdd-uav}) and a protocol module for swarm-level coordination. A simplified \acrfull{ibd} (\cref{fig:IBD-agent}) models the internal structure relevant for the protocol module. Even though flowports are named P2P\_Rx / P2P\_Tx in \cref{fig:IBD-agent}, they conceptually represent broadcast send/receive on a shared channel. Flow specifications for the flowports can be seen in \cref{fig:flowspec}.

\begin{figure}[H]
    \centering
    \includegraphics[width=\linewidth]{SD-10_IBD - Agent 1.0.png}
    \caption{\textit{(SD-10)} \acrshort{ibd} of \glspl{agent}' internal structure.}
    \label{fig:IBD-agent}
\end{figure}

The protocol module never overrides flight-critical behaviour and does not replace any of an agent's systems; The protocol module reads inputs (from the FMU and swarm communication module) and outputs high-level decisions (e.g. ''search sector S'', ''return home'', ''relay from position P'') that are then translated into flight commands by the existing control stack.

Moreover, inside each agent, the protocol module contains:
\begin{itemize}
    \item the \gls{rumourmill} logic (Origin/Source/Opinion handling),
    \item the \gls{secmarket} logic (sector cost, budgets, bidding),
    \item the Task \& Role Allocation logic (agent state machine),
    \item local information storage (StaticInfo and DynamicInfo).
\end{itemize}

The protocol module communicates with the FMU (to receive health states), the swarm communication module (to send and receive messages), and a local interface to flight/navigation systems (to request high-level actions).

\newpage

\subsection{Protocol Module Structure}
A simplified architecture of the protocol is defined by \cref{fig:bdd-protocol}, where DynamicInfo holds lists containing information about \glspl{agent} in the swarm (where the first index is always the \gls{agent} itself), information about \glspl{sector} and neighbourhoods, and StaticInfo holds enumerations of message types, health statuses, and tasks.

\begin{figure}[H]
    \centering
    \includegraphics[width=1.1\linewidth]{SD-13_BDD - Protocol 1.0.png}
    \caption{\textit{(SD-13)} \acrshort{bdd} of which main parts the protocol module contains.}
    \label{fig:bdd-protocol}
\end{figure}

\begin{figure}[H]
    \centering
    \includegraphics[width=\linewidth]{SD-16_IBD - Protocol 1.0.png}
    \caption{\textit{(SD-16)} \acrshort{ibd} of the protocol module's main parts.}
    \label{fig:ibd-protocol}
\end{figure}

\subsubsection{Sub-modules}
The protocol module consists of several submodules that relate to each other through shared memory and message flows:

\begin{itemize}
    \item Messaging, responsible for:
    \begin{itemize}
        \item[$\circ$] Receiving messages from the \acrshort{fmu}.
        \item[$\circ$] Receiving all incoming broadcasts (pulses, Origins, Sources, Opinions).
        \item[$\circ$] Mapping msgID values to internal handlers.
        \item[$\circ$] Discarding malformed or obviously stale messages.
        \item[$\circ$] Build and send outgoing messages with correct msgID and data types.
    \end{itemize}
    %\item Consensus that handles Sources and determines when consensus thresholds (e.g., two out of three sources state the same thing, or 70\% of opinions have been received) are met.
    \item Consensus, responsible for:
    \begin{itemize}
        %\item[$\circ$] Maintaining the lifecycle of the Rumour Mill.
        \item[$\circ$] Tracking active Origin rumours and their Source rumours and Opinion rumours.
        \item[$\circ$] Determining when enough Source rumours exist to interpret the event and to possibly form an Opinion rumour.
        \item[$\circ$] Tracking Opinion rumours and detect when the chosen consensus threshold $\alpha$ (70\% in examples) is reached for a given Origin.
        \item[] Consensus handles events such as:
        \item[$\circ$] ''Three Sources reached for rumour R'' $\rightarrow$ trigger to vote on truth, and create an Opinion or update local information.
        \item[$\circ$] ''Consensus threshold reached for rumour R at originator'' $\rightarrow$ trigger to make a decision.
        \item[$\circ$] Applying ownership changes when Opinions are announced.
    \end{itemize}
    %\item Task Allocation that handles bidding, sector pricing, budget tracking, and reallocations.
    \item TaskAllocation, responsible for:
    \begin{itemize}
        \item[$\circ$] Initiating bids for sectors (creating appropriate Origins).
        \item[$\circ$] Evaluating other agents’ bids when forming Sources.
        \item[] TaskAllocation uses the \gls{rumourmill} for all sector ownership changes to ensure consistent views of who owns which sector.
    \end{itemize}
    \item StateInstructions, responsible for:
    \begin{itemize}
        \item[$\circ$] Implementing the agent’s state machine (Initialising, FullyOperational, PartiallyOperational, Failed).
        \item[$\circ$] Mapping health status, sector ownership, and commands into roles (PrimarySearch, SecondaryRelay, Return, etc.).
        \item[$\circ$] Triggering sector releases and task changes when roles change.
    \end{itemize}
    \item DynamicInfo, responsible for 
    \begin{itemize}
        \item[$\circ$] Maintain AgentInfo list: members, neighbours, positions, health statuses, tasks, subjectFound, subjectLocation.
        \item[$\circ$] Maintain SectorInfo list: owner, corners, value, searched flag, cost.
        \item[$\circ$] Maintain Neighbourhood info: owner, neighbours. 
        \item[$\circ$] Maintaining the agent's budget (maxBudget, budgetRoof, currentBudget).
    \end{itemize}
    \item StaticInfo, responsible for:
    \begin{itemize}
        \item[$\circ$] Enumerations and constants such as MsgID, Task, HealthStatus, etc.
        \item[$\circ$] Possibly constant timeouts (pulse timeout, auction wait times) and global configuration.
    \end{itemize}
\end{itemize}

\subsubsection{Internal Interfaces \& Flowports}
In \cref{fig:ibd-protocolBB}, flowports named ui\_Rx and ui\_Tx are to receive broadcasts from and broadcast to SwarmInterface, flowports named P2P\_Rx and P2P\_Tx are to receive broadcasts from and broadcast to other \glspl{agent}, and the flowport named faultManUnit is for receiving messages from the \acrshort{fmu}. Flow specifications for the flowports can be seen in \cref{fig:flowspec}.
\begin{figure}[H]
    \centering
    \includegraphics[width=0.7\linewidth]{SD-15_IBD - Protocol BlackBox 1.0.png}
    \caption{\textit{(SD-15)} Blackbox view of the protocol. }
    \label{fig:ibd-protocolBB}
\end{figure}

\newpage

In \Cref{fig:flowspec}, ROr means Rumour Origin, RS means Rumour Source, and ROp means Rumour Opinion.
\begin{figure}[H]
    \centering
    \includegraphics[width=\linewidth]{SD-09_FlowSpecifications - RxTx 1.0.png}
    \caption{\textit{(SD-09)} Flow specification for flowports.}
    \label{fig:flowspec}
\end{figure}

\subsection{Communication Model}
The protocol module uses a broadcast-based wireless communication model, which means that when an agent broadcasts a message, all agents within range can receive it, assuming no interference or packet loss.

There is no concept of fixed point-to-point links or explicit routing within the protocol module. Instead, to spread information throughout the swarm, the swarm relies on broadcasts from agents, redundancy (retransmissions over time), and overlapping radio ranges.

A graphical representation of incoming and outgoing broadcasts can be seen in \cref{fig:ibd-protocolBB}, and message types can be seen in \cref{fig:flowspec}.

\vspace{0.2cm}

\subsubsection{Agent $\leftrightarrow$ Agent Communication}
Agents exchange information by broadcasting messages on a shared channel. Typical broadcast messages include:
\begin{itemize}
    \item Pulses (heartbeats).
    \item Origin rumours, such as:
    \begin{itemize}
        \item[$\circ$] \Gls{sector} bidding.
        \item[$\circ$] Sector completion.
        \item[$\circ$] Confirm \gls{subject} found.
        \item[$\circ$] Bad health.
    \end{itemize}
    \item Source rumours, such as:
    \begin{itemize}
        \item[$\circ$] Originator no longer a member of the swarm.
        \item[$\circ$] Join/rejoin attempts of an Originator.
        \item[$\circ$] Subject found.
    \end{itemize}
    \item Opinion rumours, such as:
    \begin{itemize}
        \item[$\circ$] Agent A won bid for sector S.
        \item[$\circ$] Originator should become a communication relay.
        \item[$\circ$] The number of members in the swarm.
    \end{itemize}
\end{itemize}

All messages include the sender’s AgentID, so receivers can update their local AgentInfo and justify who sent what. Any agent within range may receive the message; agents outside range may miss it.

The protocol module does not assume that every broadcast reaches every agent, instead:
\begin{itemize}
    \item Messages may be dropped due to interference or distance.
    \item Different agents may hear different subsets of transmissions.
    \item Eventual propagation is achieved because agents repeat or rebroadcast information over time and because neighbourhoods overlap spatially.
\end{itemize}

\vspace{0.2cm}

\subsubsection{Agent $\leftrightarrow$ SwarmInterface Communication}
The SwarmInterface participates in the same broadcast channel as the swarm, broadcasting and receiving messages such as:
\begin{itemize}
    \item Initialisation data (e.g. initial sector map, intended swarm size, initial neighbourhoods).
    \item High-level commands (start mission).
    \item Swarm status updates (health changes, return requests, Subject found).
    \item Perceived number of agents in swarm.
    \item Perceived number of sectors.
\end{itemize}
From the protocol module perspective, these messages are just additional broadcasts with specific msgIDs and agentID that indicate that they originate from the SwarmInterface.

Most operational commands injected by the SwarmInterface are turned into Origin rumours and processed by the \gls{rumourmill}. A small number of initialisation messages (see \Cref{subsubsec:init-event}) are treated as trusted inputs and do not trigger the rumour lifecycle.

Note that the communication between the swarm and SwarmInterface is conceptual in the current version but included in the architecture.

\subsubsection{Pulse Mechanism}
\label{subsubsec:pulse-content}
Each agent periodically broadcasts a pulse containing their AgentID, position, and timestamp of when that position was measured (see ROr\_Pulse in \cref{fig:origins}).

Any agent within range can receive the pulse and update its AgentInfo. Pulses are used to:
\begin{itemize}
    \item Detect the presence or disappearance of agents.
    %\item Agent buffer zones.
    \item Update positions.
    \item Maintain neighbourhood membership.
\end{itemize}
If an agent does not receive pulses from a neighbour within a configured timeout, it initiates a Rumour Mill event (a Source stating that the neighbour is missing). This may eventually lead to the neighbour being considered no longer a member of the swarm, its sectors being reallocated, and neighbourhoods recalculated.

\subsection{Static vs. Dynamic Information}
Each agent maintains two main categories of information:
\begin{itemize}
    \item Static information – predefined and identical for all agents, such as message IDs, task IDs, health stauses, and any constant parameters.
    \item Dynamic information – updated at runtime, including:
    \begin{itemize}
        \item[$\circ$] Known agents and their statuses (AgentInfo).
        \item[$\circ$] Neighbourhoods.
        \item[$\circ$] Known sectors, their owners, and cost (SectorInfo).
        \item[$\circ$] Budget.
        \item[$\circ$] Sector bids.
        \item[$\circ$] Current state and role.
    \end{itemize}
\end{itemize}
Static information defines the ''vocabulary'' and structure of the protocol module; Dynamic information records the current situation.

\label{subsec:stat-dyn}

%\vspace{0.2cm}
\newpage

\subsubsection{Static Information}
Static information is defined before the mission, is identical for every agent, and does not change at runtime. Static Information includes:
\begin{itemize}
    \item Enumerations of message IDs (msgID).
    \item Enumerations of task IDs (taskID).
    \item Enumerations of health statuses (supplied by the FMU).
    \item Role identifiers.
    \item Configuration parameters (e.g. pulse intervals, timeouts).
\end{itemize}

\begin{figure}[H]
    \centering
    \includegraphics[width=0.7\linewidth]{SD-14_IBD - StaticInfo 1.0.png}
    \caption{\textit{(SD-14)} Examples of what StaticInfo contains.}
    \label{fig:ibd-statinfo}
\end{figure}

\vspace{0.5cm}

\subsubsection{Dynamic Information}
Dynamic information is updated throughout the mission and stored locally on each agent, and ensures that each agent can make decisions based on its own up-to-date local view, while the \gls{rumourmill} works to keep these local views consistent across the swarm.

Dynamic information includes:
\begin{itemize}
    \item AgentInfo (\cref{fig:datatypes}) entries (per known agent):
    \begin{itemize}
        \item[$\circ$] agentID: Unique ID of agent X.
        \item[$\circ$] member: If agent X is considered a member of the swarm.
        \item[$\circ$] neighbour: If agent X is considered a neighbour by the owner of the list.
        \item[$\circ$] position: Last known latitude, longitude, and altitude of agent X (received through Pulse).
        \item[$\circ$] posTimestamp: Time that agent X broadcasted their position (received through Pulse).
        \item[$\circ$] posReceived: Time that the owner of the list received the broadcast from agent X.
        \item[$\circ$] healthStatus: Last known health status of agent X.
        \item[$\circ$] task: Last known task of agent X.
        \item[$\circ$] subjectFound: If agent X is the one that found the \gls{subject} (received by Source rumour).
        \item[$\circ$] subjectLocation: Latitude and Longitude of the Subject's position (received through Source rumour).
    \end{itemize}
    \item SectorInfo (\cref{fig:datatypes}) entries (per \gls{sector}):
    \begin{itemize}
        \item[$\circ$] sectorID: Unique ID of \gls{sector} S.
        \item[$\circ$] ownerID: ID of agent (or 0 for unassigned) that has been allocated sector S.
        \item[$\circ$] nw: Latitude and longitude of north-west corner of sector S.
        \item[$\circ$] ne: Latitude and longitude of north-east corner of sector S.
        \item[$\circ$] sw: Latitude and longitude of south-west corner of sector S.
        \item[$\circ$] se: Latitude and longitude of sector X's south-east corner.
        \item[$\circ$] value: Value based on the probability that the \gls{subject} will be found in sector S. Probability is based on \glspl{hotregion}.
        \item[$\circ$] cost: Cost of the sector.
        \item[$\circ$] searched: TRUE if ownerID has finished searching sector S, otherwise FALSE.
    \end{itemize}
    %\item Rumour state: active Origins, received Sources, formed Opinions.
    %\item Market state: current budget, outstanding bids, recently released sectors.
    %\item State machine state: current agent state (search, relay, return, etc.).
    \item Budget:
    \begin{itemize}
        %\item[$\circ$] maxBudget: Static value of the max value of the owner's budget.
        %\item[$\circ$] budgetRoof: Maximum spending allowance for the owner, based on the owner's battery level.
        \item[$\circ$] maxBudget: Maximum budget allowed for the owner, based on the owner's remaining flight time.
        \item[$\circ$] currentBudget: Current value of the owner's budget.
    \end{itemize}
    \item Neighbourhoods (per known agent):
    \begin{itemize}
        \item[$\circ$] neighbourhoodID: Unique ID of neighbourhood, corresponds to the ID of the agent that ''owns'' the neighbourhood.
        \item[$\circ$] neighbours: agent IDs that are seen as neighbours by the neighbourhood owner.
    \end{itemize}
\end{itemize}


\subsection{Information Model \& Data Types}
\Cref{fig:datatypes} depicts datatypes that can be potentially used, both for internals and broadcasts, in the protocol module. \Cref{fig:origins,fig:sources,fig:opinions} show how messages can be potentially structured.
\begin{figure}[H]
    \centering
    \includegraphics[width=0.9\linewidth]{SD-05_DataTypes 1.0.png}
    \caption{\textit{(SD-05)} Datatypes used in the system.}
    \label{fig:datatypes}
\end{figure}

\begin{figure}[H]
    \centering
    \includegraphics[width=\linewidth]{SD-06_RumourOrigins 1.0.png}
    \caption{\textit{(SD-06)} Contents of Origin Rumours.}
    \label{fig:origins}
\end{figure}

\begin{figure}[H]
    \centering
    \includegraphics[width=\linewidth]{SD-07_RumourSources 1.0.png}
    \caption{\textit{(SD-07)} Contents of Source Rumours.}
    \label{fig:sources}
\end{figure}

\begin{figure}[H]
    \centering
    \includegraphics[width=\linewidth]{SD-08_RumourOpinions 1.0.png}
    \caption{\textit{(SD-08)} Contents of Opinion Rumours.}
    \label{fig:opinions}
\end{figure}

\newpage

\subsection{External Interface Architecture}
The \gls{searchmana} interacts with the swarm through the SwarmInterface, where the core interactions are:
\begin{itemize}
    \item Input to the swarm:
    \begin{itemize}
        \item[$\circ$] Initial Search Area.
        \item[$\circ$] Hot Regions.
        \item[$\circ$] Mission start / pause / abort commands.
        \item[$\circ$] (potentially) mission updates.
    \end{itemize}
    \item Output from the swarm:
    \begin{itemize}
        \item[$\circ$] Health status summaries (e.g. ''Agent A emergency landed at location L'').
        \item[$\circ$] Return requests (battery changes).
        \item[$\circ$] \Gls{sector} completion status.
        \item[$\circ$] \Gls{subject} found.
    \end{itemize}
\end{itemize}

The External Interface Architecture defines how the protocol module interacts with external systems, primarily the SwarmInterface.

The SwarmInterface is modelled as a distinct external boundary component. This design decouples human-driven command logic from decentralised swarm behaviour and supports future expansion (e.g., GUI, map view).

Note that only the high-level interface between SwarmInterface and the protocol module is in scope; concrete message encoding and RF communication details remain out of scope for this project.

\begin{figure}[H]
    \centering
    \includegraphics[width=0.9\linewidth]{SD-11_BDD - SwarmInterface 1.0.png}
    \caption{\textit{(SD-11)} \acrshort{bdd} of a conceptual design of the SwarmInterface.}
    \label{fig:bdd-swarmint}
\end{figure}

\begin{figure}[H]
    \centering
    \includegraphics[width=0.6\linewidth]{SD-12_IBD - SwarmInterface BlackBox 1.0.png}
    \caption{\textit{(SD-12)} Blackbox view of the SwarmInterface.}
    \label{fig:ibd-swarmint}
\end{figure}

The SwarmInterface aggregates protocol module events into human-readable information, such as ''Agent A requests battery replacement'' or ''Agent B degraded'', to allow the \gls{searchmana} to prepare appropriate actions (e.g. swapping batteries when an agent returns to base).

From the protocol module perspective, the SwarmInterface provides an input channel for high-level commands and mission configuration (initial \gls{sector} map, Hot Regions, initial neighbourhoods), and an output channel for aggregated swarm status and event reports. Initial mission set up can be seen in \cref{fig:ad-swarm-reg}

\newpage

\subsection{Communication Context}
The communication context summarises how all the previously described elements fit together:
\begin{itemize}
    \item Agents communicate with one another by broadcasting messages on a shared wireless channel that carry pulses and \gls{rumourmill} traffic. 
    \item Agents communicate with the SwarmInterface on the same shared wireless channel to receive mission configurations, updates, and to report events.
\end{itemize}

Neighbourhoods provide the local structure that seeds rumour propagation, and rumours ensure that important events and decisions are eventually known by all agents. The \gls{secmarket} and Task \& Role logic run on top of this communication fabric to allocate work and determine agent behaviours.

In \cref{fig:ibd-comm-context}, even though flowports are named P2P\_Rx / P2P\_Tx in the diagram, they conceptually represent broadcast send/receive on a shared channel. Flow specifications for the flowports can be seen in \cref{fig:flowspec}.

\begin{figure}[H]
    \centering
    \includegraphics[width=1.1\linewidth]{SD-21_IBD - Communication Context 1.0.png}
    \caption{\textit{(SD-21)} \acrshort{ibd} over communication context. }
    \label{fig:ibd-comm-context}
\end{figure}
%Will insert an IBD here later. The IBD will describe the connections between SwarmInterface, Swarm, Agent, and Protocol.
%Any existing communication- or flow-oriented diagrams you have (e.g. flow specifications, block diagrams with flowports) can be referenced here as the canonical visual representation of the communication context.