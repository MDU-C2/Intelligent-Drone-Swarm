\section{Rumour Mill Protocol (Consensus Layer)}
%Rumour Mill Protocol (Consensus Layer)
\label{sec:rumour-mill}
The \gls{rumourmill} is the core consensus mechanism of the IRDS protocol module and is responsible for ensuring that all agents eventually agree on how to interpret important events (such as degraded health, \gls{sector} completion, or \gls{subject} detection) and on the decisions that follow from those events.

In this work, the combination of Origin rumours, Source rumours, Opinion rumours, and their lifecycle is referred to as the \textbf{Rumour Mill}. To the best of the project group's knowledge, using this particular rumour structure and lifecycle as a decentralised consensus mechanism for \acrshort{uav} \acrshort{sar} swarms is a novel concept introduced by this project.

The Rumour Mill is fully decentralised: Each agent executes the same logic, uses only locally stored data and received messages, and no agent is given special decision authority.

\subsection{Purpose \& Scope}
The Rumour Mill serves two main purposes:
\begin{enumerate}
    \item Shared interpretation of events
    \item[] When an event occurs (e.g. ''Agent A has degraded health''), the swarm needs a consistent answer to ''what actually happened?''.
    \item Shared decisions linked to those events
    \item[] Once the event is interpreted, the swarm needs to decide how to react (e.g. ''Agent A should switch to SecondaryRelay'').
\end{enumerate}

The Rumour Mill is used for almost all operational decisions that require swarm-wide agreement. A small set of initialisation-time inputs from the \gls{searchmana} are treated as trusted and do not use the full rumour lifecycle (see \Cref{subsubsec:init-event}).

The Rumour Mill itself does not decide how to value sectors or how to map decisions into specific roles. Instead, it provides a consensus layer that the \gls{secmarket} (\Cref{sec:sector-alloc}) and Task \& Role Allocation (\Cref{sec:task-role}) build on top of.

\subsection{Message Model \& Message IDs}
All protocol module messages are broadcast over a shared wireless channel, meaning that when an agent broadcasts, any agent within range may receive the message. Messages have a common structure that includes, at minimum:
\begin{itemize}
    \item msgID: Type identifier for the message.
    \item senderID: ID of the agent that broadcast the message.
\end{itemize}
Additional fields depend on the rumour type and event type (e.g. sectorID, healthStatus), see \cref{fig:origins,fig:sources,fig:opinions}.

The msgID field is especially important as it tells the protocol which internal handler to call and therefore what to do with the message, for example, different msgIDs may distinguish between:
\begin{itemize}
    \item Health-related Origins (e.g. degraded health such as failed LiDAR or low battery).
    \item Sector-related messages (bids, releases, and completions).
    \item Subject-related messages (found, confirmation).
    \item Search Manager commands injected through the SwarmInterface.
\end{itemize}
This design keeps the message parsing logic simple and makes it easy to extend the protocol by defining new msgID values and handlers without changing the overall architecture.

\newpage

\subsection{Rumour Types}
The Rumour Mill uses three types of rumours:
\begin{itemize}
    \item Origin Rumour: Created by an agent that directly observes or generates an event.
    \item Source Rumour: Created independently by agents that receive an Origin and interpret it.
    \item Opinion Rumour: Created by agents that have received enough Sources to summarise their view.
\end{itemize}
Together, these three types of rumours and their lifecycle implement the Rumour Mill.

\vspace{0.2cm}

\subsubsection{Origin Rumours}
An Origin rumour represents a local observation or decision made by an agent. Examples include:
\begin{itemize}
    \item ''I have degraded health'' (triggered by \acrshort{fmu}).
    \item ''I have completed \gls{sector} S.''
    \item ''\Gls{subject} found at location L.''
    \item ''I want to purchase sector S.''
    \item ''I am now the owner of sector S'' (after having received enough relevant Opinion rumours).
    %\item ''Neighbour Y is no longer a member of the swarm.''
    \item ''Search Manager issued command C.''
\end{itemize}
The agent that detects or generates the event (the originator) creates the Origin rumour and broadcasts it to its neighbours. 
%Any agent within range may receive and process it. \color{red}Conceptually, neighbour agents are those that consider the originator to be in their 1+3 neighbourhood, but physically the message is just a broadcast.\color{black}

\vspace{0.2cm}

\subsubsection{Source Rumours}
When a neighbour receives an Origin rumour, it independently evaluates the event using its local information and creates a Source rumour. A Source rumour expresses that neighbour’s interpretation of the Origin rumour, for example:
\begin{itemize}
    \item ''Agent A is no longer a member of the swarm'' (triggered by a lack of Pulse).
    \item ''I confirm that Agent A appears degraded.''
    \item ''I agree that Agent A has completed sector S.''
    \item ''I consider Agent A’s bid on sector S valid.''
    \item ''I consider Agent A’s bid on sector S invalid (conflict).''
\end{itemize}
Each Source rumour includes the ID of the originator. Source rumours are broadcast to the swarm so that other agents can accumulate multiple interpretations of the same event. An agent that holds three Source rumours can broadcast all of them to other agents.

\vspace{0.2cm}

\subsubsection{Opinion Rumours}
When an agent receives \textbf{three} Source rumours referring to the same Origin (i.e. three independent interpretations for that event), it votes internally on how to interpret the event, and then \textbf{possibly} creates an Opinion rumour.

An Opinion rumour expresses a recommended action or conclusion, for example:
\begin{itemize}
    \item ''Agent A should change to SecondaryRelay.''
    \item ''Agent A should return to base.''
    \item ''Agent A may become the owner of sector S.''
\end{itemize}
Opinion rumours are broadcast to the swarm. They summarise the local view built up from multiple Sources and are used by originators to make a final decision that is announced through a new Origin rumour.

\vspace{0.2cm}

\subsubsection{Rumour Mill as Named Mechanism}
In this document, the term \gls{rumourmill} specifically refers to the set of rumour types (Origin, Source, Opinion), and the full lifecycle described in \Cref{subsec:rumour-lifecycle}.

All consensus-related behaviour in the protocol module is built on this mechanism.

\newpage

\subsection{Rumour Lifecycle}
\label{subsec:rumour-lifecycle}
%6.4.1 Activity model for handling incoming and outgoing rumours
The lifecycle of rumours allows the swarm to converge on consistent decisions without any agent having special authority. The complete lifecycle of a rumour-based decision consists of five phases:
\begin{enumerate}
    \item Event $\rightarrow$ Origin
    \begin{enumerate}
        \item An agent detects or generates an event (e.g. FMU health change, sector completion, subject found).
        \item The agent, referred to as the originator, creates an Origin rumour and broadcasts it to agents in its neighbourhood.
        \item[$\circ$] \textit{Other agents can pick up this Origin rumour, but since all agents know all neighbourhoods they know if the Origin rumour is meant for them or not.}
    \end{enumerate}
    \item Origin $\rightarrow$ Sources
    \begin{enumerate}
        \item Agents in the originator's neighbourhood each independently interpret the Origin rumour.
        \item The neighbours individually create Source rumours and broadcast them to the swarm.
        \item[$\circ$] \textit{Agent A knows its Origin rumour has been received by ''hearing'' these Source rumours.}
    \end{enumerate}
    \item Sources $\rightarrow$ Opinions
    \begin{enumerate}
        \item Any agent, including neighbours but \textbf{not} the originator, who has received three Source rumours from three distinct agents (one of whom may be itself) considers that it has enough information to interpret the event.
        \item It makes a decision, and possibly creates an Opinion rumour and broadcasts it.
        %\item[] Note that any agent (not just the originator or its direct neighbours) can reach the ''three Sources'' threshold, come to a decision, and possibly broadcast an Opinion rumour.
    \end{enumerate}
    \item Opinions $\rightarrow$ Decision at the Originator
    \begin{enumerate}
        \item The originator collects Opinion rumours broadcast by other agents in the swarm.
        \item Once the originator has received Opinion rumours from at least a certain fraction of the swarm (the consensus threshold $\alpha$, see \Cref{subsec:consens-threshold}), it makes an internal decision.
        \item The decision may include changes such as:
        \begin{itemize}
            \item ''I will switch to SecondaryRelay.''
            \item ''I will return to base.''
            \item ''I am now owner of sector S.''
        \end{itemize}
        \item[$\circ$] \textit{The decision is made locally at the originator; no single Opinion rumour decides the outcome, and no external authority is required.}
    \end{enumerate}
    \item Decision $\rightarrow$ New Origin (Decision Announcement)
    \begin{enumerate}
        \item After deciding, and if needed, the originator broadcasts a new Origin that describes the decision (e.g. ''I will return home'', ''I have switched to SecondaryRelay'').
        \item Other agents update their local state (AgentInfo, SectorInfo, state machine, etc.) according to the decision.
        \item[$\circ$] This closes the rumour cycle. All agents now have a consistent view of the event and its outcome, subject to message delivery.
    \end{enumerate}
\end{enumerate}
\newpage
\Cref{fig:ad-rumour-mill} depicts the process of a rumour lifecycle through an Activity Diagram.
\begin{figure}[H]
    \centering
    \includegraphics[width=\linewidth]{figures/System-Design/SD-17_AD - Rumour Mill 1.0.png}
    \caption{\textit{(SD-17)} \gls{actdia} showing the sequence of actions a protocol module takes upon receiving or generating rumours.}
    \label{fig:ad-rumour-mill}
\end{figure}

\vspace{0.2cm}

\subsection{Trigger Events}
\subsubsection{Events Using the Full Rumour Lifecycle}
The \gls{rumourmill} is used for almost all operational events that require swarm-wide agreement. Typical trigger events include:
\begin{itemize}
    \item FMU-driven health events:
    \begin{itemize}
        \item[$\circ$] Degraded health (e.g. failed LiDAR).
        \item[$\circ$] Low battery (when it implies behaviour changes).
        \item[$\circ$] Critical failure (where possible).
    \end{itemize}
    \item Search-related events:
    \begin{itemize}
        \item[$\circ$] \Gls{sector} completed.
        \item[$\circ$] Sector cannot be searched (e.g. blocked, unreachable).
    \end{itemize}
    \item Subject-related events:
    \begin{itemize}
        \item[$\circ$] Subject found \textit{(request for confirmation from neighbours)}.
        %\item[$\circ$] Subject confirmation.
    \end{itemize}
    \item Agent membership events:
    \begin{itemize}
        \item[$\circ$] New agent join. 
        \item[$\circ$] Agent rejoin.
        \item[$\circ$] Agent missing \textit{(detected through pulse timeout, and is announced using a Source rumour)}.
    \end{itemize}
    \item \gls{secmarket}-related events:
    \begin{itemize}
        \item[$\circ$] Bid to purchase a sector.
        \item[$\circ$] Release of sector ownership.
    \end{itemize}
    \item Operational commands from the \gls{searchmana} (e.g. start, pause, abort, change \gls{searcharea}), as delivered via the SwarmInterface.
\end{itemize}
All of these are represented as Origins (except for the detection of a missing agent, which is handled as a Source rumour) and processed through the full rumour lifecycle.

\vspace{0.2cm}

\subsubsection{Initialisation Events Bypassing Rumour Mill}
\label{subsubsec:init-event}
Most commands are translated into \gls{rumourmill} events so that the entire swarm can respond consistently. A small set of initialisation-time commands from the Search Manager are treated as trusted configuration for the mission and do not use the full Rumour Mill process:
\begin{itemize}
    \item The initial intended number of agents in the swarm.
    \item The initial \gls{searcharea} and sector division (the sector map).
    \item The initial neighbourhoods.
\end{itemize}
After initialisation, all further operational updates and decisions use the Rumour Mill.

\subsection{Consensus Threshold \& Parameters}
\label{subsec:consens-threshold}
%There are two kinds of consensus thresholds: $\alpha$ and $\beta$.
Interpretation of and decisions based on Source rumours are made once three Source rumours have been received and an internal vote has been done (based on the principles of triple modular redundancy), this parameter can be adjusted to increase the amount of Byzantine faults that can be handled.

Agents that have broadcasted Origin rumours that require Opinion rumours must collect enough Opinion rumours for that particular Origin rumour. The required fraction of Opinion rumours is denoted as $\alpha$, where:
\begin{equation*}
    0.5 < \alpha < 1
\end{equation*}

In this document, 70\% is used as an illustrative example, meaning that the originator decides once it has received Opinions from at least 70\% of the agents it currently believes are in the swarm. This 70\% value is not fixed by the protocol, it is a tunable design parameter, and the best value will depend on:
\begin{itemize}
    \item Communication reliability and message loss.
    \item Swarm size.
    \item The expected number of faulty or unreachable agents.
    \item How quickly the swarm must respond to events.
\end{itemize}
Determining an appropriate threshold $\alpha$ for deployment is left for future work and should be investigated through simulation or analysis. The examples in this document use 70\% purely for concreteness.

\vspace{0.5cm}

\subsection{Properties of the Rumour Mill}
The \gls{rumourmill} provides several desirable properties for the protocol module:

\begin{itemize}
    \item Decentralisation:
    \begin{itemize}
        \item[$\circ$] There is no central coordinator or leader.
        \item[$\circ$] All agents run identical logic and participate symmetrically.
        \item[$\circ$] Interpretations are made locally once three Source rumours have been received.
        \item[$\circ$] Decisions are made locally by the originator based on Opinion rumours collected from the swarm.
    \end{itemize}
    \item Eventual Consistency:
    \begin{itemize}
        \item[$\circ$] Messages are broadcast in a lossy network; Not all agents hear all messages immediately.
        \item[$\circ$] However, Origins, Sources, and Opinions are rebroadcast and spread over time.
        \item[$\circ$] Provided that communication is occasionally available, all non-faulty agents eventually receive Sources or decision-Origins and converge on the same view of the event.
    \end{itemize}
    \item Fault Tolerance:% (Informal)
    \begin{itemize}
        \item[$\circ$] By requiring multiple independent Sources before interpreting and possibly broadcasting an Opinion rumour, the protocol module reduces the influence of isolated faulty interpretations.
        \item[$\circ$] By requiring a consensus threshold $\alpha$ of Opinion rumours before deciding, the originator avoids being swayed by a small number of faulty agents.
        \item[$\circ$] Neighbourhoods and overlapping broadcast ranges ensure that events are cross-checked by multiple agents.
        \item[$\circ$] A more detailed fault-tolerance analysis (including Byzantine assumptions and neighbourhood constraints) is provided in \Cref{sec:safety-reliability}.
    \end{itemize}
    \item Extensibility and Implementation Friendliness:
    \begin{itemize}
        \item[$\circ$] New event types can be added by defining new msgID values and handlers without changing the basic Origin/Source/Opinion structure.
        \item[$\circ$] Messages remain small and simple, suitable for microcontrollers and limited-bandwidth radio links.
        \item[$\circ$] The algorithm uses only basic counting and storage, making it practical for embedded systems.
    \end{itemize}
\end{itemize}

\subsection{Example Rumour Cycles}
This section gives brief examples of how the Rumour Mill works in typical situations.

These scenarios show how the Rumour Mill links local observations to swarm-wide, consistent decisions without a central controller.

\vspace{0.2cm}

\subsubsection{Degraded Health Event}
\begin{enumerate}
    \item \acrshort{fmu} on agent A detects a bad sensor (e.g., bad LiDAR).
    \item A releases its sectors and broadcasts an Origin: ''I (A) am degraded: bad LiDAR.''
    \item Neighbours of A interpret it and broadcast Sources confirming the degraded state.
    \item Any agent (including neighbours of A) that receives three Sources updates A's sectors to unassigned, and creates an Opinion rumour (e.g. ''A should release its sectors and become SecondaryRelay'').
    \item A collects Opinion rumours; Once Opinion rumours from $\geq \alpha$ fraction of agents have arrived, A internally decides to switch to SecondaryRelay.
    \item A broadcasts a decision-Origin announcing its new role.
    \item The \gls{secmarket} (\Cref{sec:sector-alloc}) reassigns A’s sectors to other agents through bidding.
\end{enumerate}

\vspace{0.2cm}

\subsubsection{Subject Found Event}
\begin{enumerate}
    \item Agent A detects the \gls{subject} at location L and broadcasts an Origin: ''Subject found at L.''
    \item Neighbours of A receive the Origin rumour and those who are \textbf{fully functional} go to L to confirm that the subject is at L.
    \item Confirming neighbours (i.e. fully functional agents) broadcast Source rumours based on their findings (e.g. ''Subject not found at L'' or ''Subject found at L'').
    \item Once any agent has three Sources (including originator and neighbours), it interprets the event by either continuing its current task (''Subject not found at L'') or updating their state machines to enter roles appropriate for mission completion (Returning).
\end{enumerate}


