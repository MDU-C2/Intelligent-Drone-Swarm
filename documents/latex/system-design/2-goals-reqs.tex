\section{Goals \& Requirements Overview}
\label{sec:goals}
This section summarises the goals and requirements that drive the IRDS protocol module. Detailed specifications are kept in four artefacts (goals, drone\_swarm\_requirements, system\_requirements, subsystem\_requirements in the project's requirements database) \cite{rm}; only the most relevant elements are repeated here. The aim is to show what the system must achieve at the project, swarm, protocol, and sub-module levels, and how these requirements shape the architectural and behavioural design choices in the rest of this document.

\subsection{Goals \& Requirements Overview}
The goals and requirements are maintained in the following main artefacts in the requirements database \cite{database}:
\begin{itemize}
    \item goals - project goals derived from stakeholder needs and the KAOS method.
    \item drone\_swarm\_requirements - swarm-level behavioural and safety requirements.
    \item system\_requirements - requirements allocated to the IRDS protocol module as a system.
    \item subsystem\_requirements - requirements allocated to the internal sub-modules of the protocol.
\end{itemize}

\subsection{Project Goals}
The project goals in \textit{goals} capture what the IRDS System of Systems (SoS) is intended to achieve at a high level. Key examples include:
\begin{itemize}
    \item G-01 - Assign an appropriate task to each agent at the start of the mission.
    \item G-02 - When an agent fails, automatically reassign its tasks to other functional agents when resources permit.
    \item G-03 - Maintain correct swarm behaviour despite failure or inconsistent output from single agents.
    \item G-04 / G-05 - Reach collective decisions for tasks that require coordination and ensure that the search area is distributed among agents.
    \item G-08 / G-09 - Maintain mission continuity and recover from agent failures under the chosen protocol.
    \item G-11 / SG-01 - Allow agents to make individual decisions when appropriate while maintaining staff safety in all operational phases.
\end{itemize}
The IRDS protocol module contributes to these goals primarily through its three main functional areas: Rumour Mill (decentralised consensus on mission state and events), Sector Allocation Market (distribution of search sectors based on mission priorities and agent state), and Task \& Role Allocation (mapping sector ownership and health status into concrete agent behaviour).

\vspace{0.2cm}

\subsubsection{Requirements Structure}
%\subsection{Swarm-level Requirements}
%\subsection{Protocol Module Requirements}
%\subsection{Internal Sub-module Requirements}
The requirements are organised in three main layers that correspond to different responsibility levels in the architecture:
\begin{itemize}
    \item Swarm-level requirements (drone\_swarm\_requirements)
    \item[] These requirements (e.g. SW-01-SW-07, SR-02-SR-07) describe how the swarm as a whole shall behave. Examples include:
    \begin{itemize}
        \item[$\circ$] SW-01 - Assigning tasks to all agents when a mission starts.
        \item[$\circ$] SW-02 / SW-03 - Reallocating work when agents degrade or fail.
        \item[$\circ$] SW-04 / SW-05 - Employing a consensus mechanism that maintains correct behaviour even when at least one agent provides incorrect data.
        \item[$\circ$] SW-06 - Coordinating agents so that they cover different parts of the Search Area.
        \item[$\circ$] SW-07 - Allowing agents to make independent local decisions when immediate action is required.
        \item[$\circ$] SR-02-SR-07 - Maintaining safety distances and buffer zones around agents and staff .
    \end{itemize}
    \item System-level protocol requirements (system\_requirements)
    \item[] These requirements (PM-01 - PM-21) define what the IRDS protocol module as a system shall do, for example:
    \begin{itemize}
        \item[$\circ$] PM-01 - Define procedures to keep all agents aware of the prevailing mission state.
        \item[$\circ$] PM-02 / PM-03 / PM-06 - Define procedures for sharing the mission plan and assigning tasks to agents based on the plan and their capabilities.
        \item[$\circ$] PM-04, PM-05 - Define how agents communicate their capabilities and the message formats used for this communication.
    \end{itemize}
    \item Subsystem-level protocol requirements (subsystem\_requirements)
    \item[] These requirements are allocated to internal sub-modules of the protocol (Consensus, Information, Messaging, State Instruction, etc.). Examples include:
    \begin{itemize}
        \item[$\circ$] CON-01 / CON-02 - the Consensus sub-module shall specify how agents confirm agreement on the prevailing mission state and how they reach a common understanding of the mission plan.
        \item[$\circ$] INF-01-INF-03 - the Information sub-module shall define the content and structure of mission state data and the mission plan, and how agents retrieve this information.
        \item[$\circ$] MSG-01 / MSG-02 - the Messaging sub-module shall define communication rules for transmitting mission state and mission plan data between agents.
        \item[$\circ$] SIS-01 - the State Instruction sub-module shall define how instructions are derived from the current mission state and issued to agents.
    \end{itemize}
\end{itemize}
In the architectural view adopted in this document, the IRDS protocol module is primarily responsible for satisfying the system-level and subsystem-level requirements, and for contributing to the swarm-level requirements that concern coordination, consensus, and task allocation. Low-level flight behaviour, obstacle avoidance, and detailed motion control requirements are delegated to the UAV flight systems (e.g. FCU, ODU, CAU), which are treated as external components.

\vspace{0.2cm}

\subsubsection{Traceability}
A mapping from goals and requirements to system design elements (SD-xx) is maintained in the project's requirements database \cite{database}, and a mapping of requirements to sections in this document is maintained in \cite{trace-design}.

