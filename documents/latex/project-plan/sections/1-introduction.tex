\section{Introduction}
This Project Plan provides team members and stakeholders with an overview of the \acrfull{irds} project. This plan explains the background, purpose and objectives of the project and introduces the overall management approach adopted by the team. In addition, it defines how the project will be structured and controlled to ensure that activities, roles, and deliverables align with the project’s dependability and safety goals.

%\vspace{1cm}

\subsection{Background}
The original idea behind this project is the development of a fail-operational \acrfull{uav} swarm designed for \acrfull{sar} missions. Such a swarm can also be applied in both defence and civilian operations, where the ability to operate autonomously is crucial. This concept is based on the principles of collective and individual behaviour, enabling \acrshort{uav}s to function independently while also coordinating as a group.

A key aspect of this concept is autonomous navigation and control in dynamic and uncertain environments, where reliability and adaptability are essential. Previous research has introduced a novel conceptual architecture that bridges the critical gap between single-\acrshort{uav} fault tolerance and collective swarm fault tolerance. Although fault tolerance traditionally ensures that an individual \acrshort{uav} can withstand failures, fault tolerance at the swarm level emphasises the ability of the group as a whole to recover and continue operating even after losing an entire \acrshort{uav}.

This dual focus on individual and collective dependability forms the foundation for the development of \acrshort{uav} swarms capable of performing effectively in challenging real-world scenarios.

The purpose of this project is to implement a swarm-level coordination logic that will ensure that a \acrshort{sar} mission can continue with maximum efficiency, even when individual \acrshort{uav}s are compromised.

%\vspace{1cm}

\subsection{Document Purpose}
The purpose of this plan is to define how the project will be managed and executed throughout its lifecycle. This document establishes the framework for planning, coordinating, communicating, and controlling project activities. In addition, it provides guidance for all team members and ensures that the work is carried out in a consistent way with the project’s dependability, safety, and quality objectives.

\subsection{Document Scope}
This document defines the overall management framework for the \acrshort{irds} project. It integrates and governs activities described in associated management plans (see \cref{section:related-docs} \nameref{section:related-docs}) to ensure consistency and coherence across all processes.

This plan applies to the entire project lifecycle and specifies how the project is organised, scheduled, and controlled. In addition, it establishes the interrelations between management roles, artefacts, and deliverables.

Specifically, this document defines the following:
\begin{itemize}
    \item Project context, including stakeholders, objectives, and external dependencies.
    \item Scope of the project and deliverables to be produced by the project team.
    \item Methodology, tools and lifecycle approach used to execute and monitor the project.
    \item Organisation, including project roles, communication channels, and responsibilities.
    \item Integration of subordinate management plans to ensure consistent execution of safety, quality, requirements, configuration, and verification processes.
    \item Schedule, milestones, and risk management structure that guide the execution of the project.
\end{itemize}

This plan serves as the top-level governing document for all activities. All management plans, artefacts, and deliverables produced during the project shall conform to the framework and principles established in this document.

\newpage
\subsection{Objectives}
\begin{itemize}
    \item Develop a decentralised replanning \gls{protocol} that enables a UAV swarm to adapt collectively when individual agents experience degraded health
    \item Design and implement a safe consensus mechanism that allows all agents to agree on a new mission plan after fault detection
    \item Create adaptive task reallocation logic to redistribute tasks from compromised agents to healthy agents, and assign supportive roles to compromised agents
    \item Validate through simulation with fault injection, measuring improvements in mission continuity, resilience, and efficiency
\end{itemize}

\vspace{0.5cm}
