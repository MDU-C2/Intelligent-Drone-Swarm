\chapter*{Project Management Approach}
\addcontentsline{toc}{chapter}{Project Management Approach} % adds it to the TOC

\section{Governance Structure}
governance structure in a project plan is about how the project will be directed, controlled, and held accountable*

It defines the decision-making framework and the people or groups responsible for oversight.

Key elements of project governance structure

1. Roles \& responsibilities
Who’s sponsoring the project, who’s managing it, and who’s on the steering or oversight committee.
Example: Project Sponsor, Project Manager, Steering Committee, Workstream Leads.

2. Decision-making hierarchy
How major decisions will be made and escalated.
Example: Steering committee approves budget changes; project manager approves day-to-day scope changes.

3. Accountability mechanisms
Who is ultimately accountable for project success and how performance will be tracked.
Example: Sponsor ensures alignment with business goals; PM ensures delivery within constraints.

4. Meeting and reporting cadence
When and how governance bodies will meet and review progress.
Example: Monthly steering committee meetings, weekly PMO reports.

5. Escalation paths
Where issues go if they can’t be resolved at the team level.
Example: Risks unresolved by workstream lead → Project Manager → Steering Committee.

Think of governance structure as the rules of the game for the project: who decides, who approves, who is accountable, and how decisions are escalated.
