\section{Project Organisation \& Communication}

\subsection{Project Roles}
\begin{table}[H]
    \centering
    \begin{tabular}{|l|c|c|l|}
        \hline
        \textbf{Role} & \textbf{Responsibilities} & \textbf{Authority} & \textbf{Dependability Focus}\\
        \hline
        Chief Engineer & \makecell{Lead overall project execution;\\ integrate all activities;\\ maintain schedule;\\ make technical trade-off decisions;\\ ensure dependability requirements\\ are met} & \makecell{Approve major\\ technical changes;\\ assign responsibilities\\ and resources; sign off\\ on final system-level\\ deliverables} & \makecell{Ensures overall system\\ reliability, safety, and\\ robustness by integrating\\ dependability considerations\\ across all activities}\\
        \hline
        \makecell{Requirements\\ Manager} & \makecell{Capture, analyse, prioritise, and track\\ requirements; ensure clarity, feasibility,\\ and verifiability; maintain traceability;\\ coordinate with Safety, V\&V,\\ and Q\&C Managers} & \makecell{Approve changes to\\ requirements baseline;\\ reject incomplete or\\ untestable requirements;\\ request clarifications\\ from team members} & \makecell{Ensures that dependability\\ requirements (safety,\\ reliability, availability)\\ are correctly captured and\\ traceable throughout the project}\\
        \hline
        \makecell{Validation \&\\ Verification\\ Manager} & \makecell{Develop and execute verification\\ and validation plans; ensure traceability\\ between requirements, design, and\\ V\&V results; document outcomes;\\ support dependability assessments} & \makecell{Approve completion\\ of V\&V activities;\\ reject deliverables failing\\ verification/validation;\\ recommend corrective actions} & \makecell{Confirms that system\\ functions as intended\\ and meets dependability\\ requirements through\\ systematic verification\\ and validation}\\
        \hline
        \makecell{Safety Manager} & \makecell{Identify hazards and perform risk\\ assessments; define and maintain safety\\ requirements; ensure compliance with\\ safety standards; coordinate with\\ Requirements and V\&V Managers} & \makecell{Approve safety-related\\ design changes; halt\\ project activities violating\\ safety requirements;\\ require corrective actions\\ for hazards} & \makecell{Protects system and\\ users by mitigating\\ risks and ensuring all\\ safety-critical aspects\\ of dependability are\\ addressed} \\
        \hline
        \makecell{Quality \&\\ Configuration\\ Manager} & \makecell{Define quality standards;\\ ensure deliverables meet quality criteria;\\ manage configuration control and change\\ management; monitor process adherence;\\ maintain project documentation} & \makecell{Approve changes to\\ controlled documents\\ and baselines;\\ enforce quality standards;\\ reject outputs not meeting\\ quality criteria} & \makecell{Ensures consistent\\ quality and integrity\\ of project outputs,\\ supporting reliability,\\ maintainability, and\\ traceability of the system}\\
        \hline
    \end{tabular}
    \caption{Project roles.}
    \label{tab:project-roles}
\end{table}

\subsection{Communication}
The types of communication that will be used during the project will be: 
%During the project the team will have daily and weekly scrum meetings:
\begin{itemize}
    \item Daily in-person scrum meetings with a duration of approximately 15 minutes that handle:
    \begin{itemize}
        \item[$\circ$] \textbf{Yesterday’s achievements:} Describe what you were able to do yesterday.
        \item[$\circ$] \textbf{Today’s achievements:} Describe what you intend to do today.
        \item[$\circ$] \textbf{Blockers:} Describe anything that you need answering or unblocking.
    \end{itemize}
    \item Weekly in-person scrum meetings with a duration of approximately one hour that will handle: 
    \begin{itemize}
        \item[$\circ$] \textbf{Progress:} Review all activities and tasks done.
        \item[$\circ$] \textbf{Slowed down:} Activities and tasks that have not made the progress the team was expecting.
        \item[$\circ$] \textbf{Stopped:} Activities and tasks stopped in their tracks.
    \end{itemize}
    \item Discord for written communication, with text channels:
    \begin{itemize}
        \item[$\circ$] \textbf{general}: For any communication about the project that does not belong in any other channel.
        \item[$\circ$] \textbf{digital-spaces}: To keep track of which digital spaces we use and direct links to them.
        \item[$\circ$] \textbf{github}: For announcing updates about the project's GitHub repository and its branches.
        \item[$\circ$] \textbf{sharepoint}: To announce when files have been uploaded to SharePoint.
        \item[$\circ$] \textbf{latex}: For LaTeX templates, acronyms used, etc.
        \item[$\circ$] \textbf{irrelevant}: For anything else.
    \end{itemize}
\end{itemize}

\subsection{Task Management}
Managing, assigning, and breaking down activities and tasks shall be done in Jira to make it easier to track progress and to get a graphical representation of the project's timeline and progress. 

In Jira, the progress of tasks and activities will be kept track of by using a kanban approach (figure \ref{fig:kanban}), where: 
\begin{itemize}
    \item \textbf{TO DO} is for tasks that have not yet been started,
    \begin{itemize}
        \item[$\circ$] or for tasks with outputs that did not get approval from a review and \textbf{\textit{need}} adjustments.
    \end{itemize}
    \item \textbf{IN PROGRESS} is for tasks that are currently being worked on,
    \begin{itemize}
        \item[$\circ$] or for tasks with outputs that did not get approval from a review and are \textbf{\textit{getting}} adjustments.
    \end{itemize}
    \item \textbf{TO BE REVIEWED} is for tasks that have outputs that need to be reviewed,
    \begin{itemize}
        \item[$\circ$] or for tasks with outputs that did not get approval from a review, have been adjusted, and are then \textbf{\textit{waiting for additional}} reviewing.
    \end{itemize}
    \item \textbf{IN REVIEW} is for tasks with outputs that are currently being reviewed
    \begin{itemize}
        \item[$\circ$] or for tasks with outputs that did not get approval from a review, have been adjusted, and are then \textbf{\textit{getting additional}} reviewing.
    \end{itemize}
    \item \textbf{DONE} is for tasks that are finished and tasks with outputs that have been approved through a review.
\end{itemize}

\begin{figure}[H]
    \centering
    \includegraphics[width=0.6\textwidth]{figures/kanban.png}
    \caption{Kanban board.}
    \label{fig:kanban}
\end{figure}

\subsection{File Management}
\begin{itemize}
    \item The project's GitHub repository stores source code, approved documents, and the project's requirements database.
    \item SharePoint is used as storage for files that are not for public use, documents waiting for review, and protocol templates.
\end{itemize}

\newpage
\subsection{Git}
\subsubsection{Pull Requests}
\begin{itemize}
    \item All changes to the main branch must go through a \acrfull{pr}.
    \item A \acrshort{pr} may only be merged if:
    \begin{itemize}
        \item[$\circ$] It has been approved by the \acrlong{ce}.
        \item[$\circ$] It has been approved by at least one other reviewer who is not the author of the change.
    \end{itemize}
    \item All \acrshort{pr}s must include:
    \begin{itemize}
        \item[$\circ$] A clear description of the change and its purpose.
        \item[$\circ$] References to related activities or tasks.
        %\item[$\circ$] Evidence that the code has been tested (e.g., test results, screenshots, or logs).
    \end{itemize}
    \item The author of the \acrshort{pr} is responsible for resolving all review comments before merging.
    \item Squash merges are preferred to keep the commit history clean, unless there is a reason to preserve individual commits.
\end{itemize}

\subsubsection{Branching Strategy}
\begin{itemize}
    \item The main branch represents the public-ready state of the project and is visible to the public.
    \item All changes must be done in feature branches.
    \begin{itemize}
        \item[$\circ$] Naming convention: feature/<short-description> (e.g., feature/id-list).
    \end{itemize}
    \item Optionally, a develop branch may be used to integrate multiple changes before merging into main.
    \begin{itemize}
        \item[$\circ$] This branch represents the ''next public-ready candidate.''
        \item[$\circ$] There can only be one develop branch at a time.
    \end{itemize}
    \item Branches should be deleted after merging to avoid clutter.
\end{itemize}

\subsubsection{Commit Standards}
\begin{itemize}
    \item Commits should be small, focused, and logically grouped.
    \item Commit messages must:
    \begin{itemize}
        \item[$\circ$] Use imperative mood (e.g., ''Add login validation'' instead of ''Added login validation'').
        \item[$\circ$] Clearly describe the purpose of the change.
        %\item[$\circ$] Reference issue numbers when applicable (e.g., Fix #123).
    \end{itemize}
\end{itemize}








