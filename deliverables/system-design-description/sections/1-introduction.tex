% !TeX root = ../main.tex
\section{Introduction}
\label{sec:intro}
The \acrfull{irds} protocol module is a decentralised coordination and replanning system for \acrfull{uav} swarms in \acrfull{sar} missions. Each UAV runs an instance of the protocol, allowing the swarm to continue operating when individual agents degrade or fail, without relying on a central controller. This document gives a conceptual design of the protocol module: it describes the context and goals of the system, the architectural structure, the Rumour Mill consensus mechanism, the Sector Allocation Market, the Agent Task \& Role Allocation layer, and the safety and fault-tolerance model.

\subsection{Purpose}
This document describes the conceptual design of the IRDS protocol module and serves as a reusable technical specification for future work. It explains:
\begin{itemize}
    \item How the system is architecturally structured.
    \item How information is exchanged and agreed upon using the \gls{rumourmill}.
    \item How search work is distributed using the \gls{secmarket}.
    \item How individual agents choose roles and behaviours using Task \& Role Allocation.
    \item How safety and fault tolerance are achieved at the swarm level.
\end{itemize}


\subsection{Intended Audience}
This document is intended for:
\begin{itemize}
    \item Project groups continuing the development of the IRDS.
    \item Researchers studying decentralised UAV coordination.
    \item Designers developing UAV swarm-level logic, interfaces, or simulations.
    \item Supervisors and examiners evaluating design decisions against course requirements.
\end{itemize}
The reader is expected to have basic familiarity with UAV systems, distributed coordination concepts, and general principles of software and system architecture.

\subsection{System Overview}
The IRDS protocol module is a decentralised coordination and replanning system for UAV swarms performing Search and Rescue (SAR) missions. Each UAV hosts an instance of the protocol module, which enables the swarm to assign work, react to degraded agent health, and maintain mission continuity without relying on a single central controller. The protocol interacts with a SwarmInterface component that exposes high-level commands from the Search Manager (such as mission start, search area, and hot regions) and reports back swarm status and key events.

Conceptually, the protocol module is organised into three functional layers. The Rumour Mill layer provides decentralised consensus on mission state and significant events. The Sector Allocation Market layer uses this shared state to distribute search sectors and rebalance work when agents degrade or fail. The Agent Task \& Role Allocation layer translates sector ownership and agent health into concrete roles and tasks that each UAV executes through its onboard flight systems. This document specifies the behaviour and architecture of the protocol module at this conceptual level, while treating low-level flight control, detailed sensing, and physical communication mechanisms as external to the system.

\newpage

\subsection{Novel Contributions}
The IRDS protocol module provides the following main contributions:
\begin{itemize}
    \item A decentralised protocol architecture for UAV swarms that maintains mission continuity without a single point of failure.
    \item A rumour-based consensus mechanism tailored to SAR missions, using neighbourhoods and simple message types to achieve eventual consistency.
    \item A sector allocation market that distributes and rebalances search work based on agent health, workload, and sector value.
    \item An agent-level task and role model that links swarm-level decisions to concrete UAV behaviour.
    \item A safety and fault-tolerance model that combines local flight-system safeguards with swarm-level reasoning about degraded or faulty agents.
\end{itemize}

\subsection{Document Structure}
The remainder of this document is structured as follows:
\begin{itemize}
    \item \Cref{sec:goals} - \nameref{sec:goals}
    \item[] Summarises the project goals and layered requirements (swarm, protocol, and sub-modules) and shows how they guide the design.
    \item \Cref{sec:system-context} - \nameref{sec:system-context}
    \item[] Describes the SAR environment, external actors (\gls{searchmana}, SwarmInterface), system boundaries, assumptions, and constraints.
    \item \Cref{sec:system-overview} - \nameref{sec:system-overview}
    \item[] Summarises the main functional areas (Rumour Mill, Sector Allocation Market, Task \& Role allocation) and presents high-level swarm workflows and scenarios.
    \item \Cref{sec:system-architecture} - \nameref{sec:system-architecture}
    \item[] Describes the structural decomposition of the IRDS system, including agent architecture, neighbourhood structure, communication model, and data structures.
    \item \Cref{sec:rumour-mill} - \nameref{sec:rumour-mill}
    \item[] Defines the rumour message types, lifecycle, trigger events, and consensus thresholds. This section formalises what is meant by ''Rumour Mill''.
    \item \Cref{sec:sector-alloc} - \nameref{sec:sector-alloc}
    \item[] Explains how sectors are valued, how agents budget and bid for sectors, and how sector ownership is updated via the Rumour Mill.
    \item \Cref{sec:task-role} - \nameref{sec:task-role}
    \item[] Describes how agents choose roles (search, relay, return, etc.) based on health status, sector ownership, and commands, and how this is implemented as a state machine.
    \item \cref{sec:safety-reliability} - \nameref{sec:safety-reliability}
    \item[] Presents fault model, pulse-based failure detection, Byzantine fault tolerance assumptions, and safety-related behaviour.
    \item \Cref{sec:future-work} - \nameref{sec:future-work}
    \item[] Outlines recommended directions for further development and research, such as implementation, security, and large-scale validation.
    \item Appendix
    \item[] Provides supporting details such as hardware diagrams.
\end{itemize}