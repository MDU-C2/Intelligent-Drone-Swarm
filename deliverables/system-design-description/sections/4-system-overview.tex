\section{System Overview}
\label{sec:system-overview}
%The protocol module enables a \acrshort{uav} swarm to autonomously coordinate, share information, and re-plan mission execution in response to agent health degradation or other operational events.
This section provides a high-level behavioural overview of the IRDS protocol module and how it coordinates the swarm during a mission. It explains the design philosophy behind the protocol, introduces the three core functional layers (Rumour Mill, Sector Allocation Market, and Agent Task \& Role Allocation), and describes the overall operational workflow at an intuitive level. Detailed structural decomposition and low-level protocol mechanics are intentionally deferred to Sections \ref{sec:system-architecture} to \ref{sec:task-role}; the focus here is on how the main concepts fit together to produce the desired swarm behaviour.

\subsection{Design Philosophy}
The protocol module is designed around three guiding principles: Decentralisation, collective adaptation, and fault tolerance.

\vspace{0.2cm}

\subsubsection{Decentralisation}
%The swarm must continue to operate safely and effectively without a central controller. Decisions emerge from interactions between agents, allowing the system to remain robust even when individual agents fail or communication is intermittent.

The protocol module is designed to operate without any central controller or privileged ''mother drone''. Instead, every agent executes the same protocol module, makes decisions based on its own state and received messages, and participates equally in consensus and allocation. Those factors remove single points of failure: If any specific agent fails, the rest of the swarm continues to operate and make decisions.

\vspace{0.2cm}

\subsubsection{Collective Adaptation}
The swarm must adapt when conditions change. Examples include:
\begin{itemize}
    \item An agent’s health degrading.
    \item An agent running low on battery.
    \item Communication loss.
    \item Discovery of the \Gls{subject}.
    \item Changes in the Search Area commanded by the \gls{searchmana}.
\end{itemize}

Rather than adapting locally in an \gls{adhoc} way, the protocol module coordinates reactions collectively; Events are shared via rumours, decisions are agreed via the Rumour Mill, \glspl{sector} are reallocated through the \gls{secmarket}, and agents update their roles accordingly.

\vspace{0.2cm}

\subsubsection{Fault Tolerance}
Using neighbourhoods, the swarm can tolerate Byzantine faults within each neighbourhood to ensure that no single faulty agent can corrupt group decisions or destabilise the mission.

%The protocol module assumes that agents may crash or disappear, messages may be lost, and that some agents may behave arbitrarily (Byzantine faults).

The protocol module uses the following to tolerate Byzantine faults:
\begin{itemize}
    \item Neighbourhoods (each agent + its three closest neighbours) as basic rumour propagation groups.
    \item A rumour lifecycle that
    \begin{itemize}
        \item[$\circ$] requires three independent Source rumours before an agent A interprets the event from agent B through internal voting, and
        \item[$\circ$] after event interpretation, agent A possibly broadcasts an Opinion, where
        \item[$\circ$] agent B (the originator) receives Opinions up to a certain consensus threshold $\alpha$ before making a final decision.
    \end{itemize}
\end{itemize}
This design localises the impact of faulty agents and supports safe mission continuation despite individual failures.

\newpage
\subsection{Core Functional Layers}
The protocol can be decomposed into several core functional layers:

\vspace{0.2cm}

\subsubsection{Rumour Mill (Consensus Layer)}
The Rumour Mill is the decentralised consensus mechanism used by the swarm to reach shared agreement about mission events (e.g. “Agent A is degraded”, “Sector S is completed”) and the decisions that follow from them (e.g. “Agent A should become relay”, “Agent A owns sector S”). Each agent broadcasts and interprets rumours about events, and converges to decisions based on multiple, independent confirmations from its neighbourhood. The full rumour types, message formats, and lifecycle are described in \Cref{sec:rumour-mill}.

\vspace{0.2cm}

\subsubsection{Sector Allocation Market (Workload Layer)}
The Sector Allocation Market decides which agent is responsible for which sectors. It uses information about sector value and location, agent health and workload, and current sector ownership to distribute work and rebalance it when agents degrade or fail. The market operates on top of the Rumour Mill, so all changes in sector ownership are agreed via rumour-based consensus. The detailed market model and bidding process are described in \Cref{sec:sector-alloc}.

\vspace{0.2cm}

\subsubsection{Agent Task \& Role Allocation (Behaviour Layer)}
The Agent Task \& Role Allocation layer determines what each agent actually does at any given time. It maps sector ownership (from the Sector Allocation Market), health status (from the FMU), and Search Manager commands (via the SwarmInterface) into roles and tasks such as searcher, relay, return, or failed. These roles drive the agent’s state machine and resulting flight behaviour. The state machine and role logic are described in detail in \Cref{sec:task-role}.

\subsection{High-Level Operational Workflow}
%This loop runs continuously for the duration of the mission, allowing the swarm to adapt to failures and changes.
At a high level, the protocol module operates according to the following recurring loop:
\begin{enumerate}
    \item Event occurs, such as:
    \begin{itemize}
        \item \acrshort{fmu} detects health degradation.
        \item A sector is completed.
        \item A new agent joins.
        \item A Search Manager command is issued.
        \item A pulse is missed.
        \item The \gls{subject} is found.
    \end{itemize}
    \item Rumour Mill processes the event:
    \begin{itemize}
        \item The event becomes an Origin rumour.
        \item Neighbours create Source rumours interpreting the event.
        \item Agents that have received three Source rumours interpret the event through internal voting.
        \item Agents possibly create Opinion rumours after voting.
        \item The originator collects Opinions until the consensus threshold $\alpha$ is reached and decides.
        \item The decision is announced through a new Origin rumour.
    \end{itemize}
    \item Sector Allocation Market updates responsibilities:
    \begin{itemize}
        \item Degraded agents release sectors.
        \item Healthy agents bid on available sectors.
        \item Ownership changes are confirmed through the Rumour Mill.
    \end{itemize}
    \item Agent Task \& Role Allocation updates behaviours:
    \begin{itemize}
        \item Based on health, sectors, and decisions, each agent updates its state and role.
        \item Some agents continue searching, some become relays, some return to base, etc.
    \end{itemize}
    \item Swarm continues the mission:
    \begin{itemize}
        \item Updated roles and sector allocations produce an adjusted search pattern.
        \item The cycle repeats as new events occur.
    \end{itemize}
\end{enumerate}

\subsection{Representative Operational Scenarios}
The following scenarios illustrate how the layers work together.

\vspace{0.2cm}

\subsubsection{Swarm Registration at Initial Start-Up}
\Cref{fig:ad-swarm-reg} does not cover agents joining the swarm after mission start.
\begin{figure}[H]
    \centering
    \includegraphics[width=\linewidth]{figures/System-Design/SD-03_AD - Swarm Registration 2.0.png}
    \caption{\textit{(SD-03)} \gls{actdia} of swarm registration at initial start-up.}
    \label{fig:ad-swarm-reg}
\end{figure}

\subsubsection{Search Area Division}
\begin{figure}[H]
    \centering
    \includegraphics[width=0.7\linewidth]{figures/System-Design/SD-04_AD - Search Area Division 1.0.png}
    \caption{\textit{(SD-04)} \gls{actdia} of dividing a \gls{searcharea} into \glspl{sector}.}
    \label{fig:ad-searcharea-division}
\end{figure}

\subsubsection{Initial Sectors Assigned}
\begin{figure}[H]
    \centering
    \includegraphics[width=\linewidth]{figures/System-Design/SD-20_AD - Initial Sectors 1.0.png}
    \caption{\textit{(SD-20)} \gls{actdia} of assigning initial \glspl{sector}.}
    \label{fig:ad-initial-sectors}
\end{figure}


\subsubsection{New Agent Join After Mission Start}
\begin{figure}[H]
    \centering
    \includegraphics[width=1.1\linewidth]{figures/System-Design/SD-22_AD - Agent Join After Mission Start 1.0.png}
    \caption{\textit{(SD-22)} \gls{actdia} showing the process of agents joining after mission start.}
    \label{fig:ad-new-agent}
\end{figure}

\newpage

\subsubsection{Agent Rejoin}
%Agent re-joining after temporary loss
\color{red}\color{black}
\begin{figure}[H]
    \centering
    \includegraphics[width=0.4\linewidth]{figures/System-Design/SD-23_AD - Agent Rejoin 1.0.png}
    \caption{\textit{(SD-23)} \gls{actdia} showing the process of agents re-joining the swarm.}
    \label{fig:ad-agent-rejoin}
\end{figure}


\subsubsection{Agent Health Degrades}
\begin{enumerate}
    \item FMU on Agent A reports degraded health (e.g. sensor fault).
    \item A releases its \glspl{sector} and broadcasts an Origin: ''I am degraded (bad\_sensor)''.
    \item Neighbours interpret the Origin rumour and broadcast Source rumours about the degraded state.
    \item Agents that have received three Source rumours interpret the event through internal voting.
    \item Agents broadcast Opinion rumours; A receives enough Opinions and decides to switch to a secondary role.
    \item A broadcasts a new Origin announcing its decision.
    \item Its sectors re-enter the \gls{secmarket}; healthy agents bid and take over.
    \item A moves to a secondary role, such as a communication relay.
\end{enumerate}

\subsubsection{Sector Completed}
\begin{enumerate}
    \item Agent A finishes searching sector S.
    \item A creates an Origin rumour: ''I (A) have completed sector S.''
    \item Neighbours interpret the Origin rumour and broadcast Source rumours confirming that S is completed.
    \item Agents that have received three Source rumours interpret the event through internal voting.
    \item Agents update their SectorInfo.
    \item The budget of A increases by the cost of S, and A bids for a new sector.
\end{enumerate}

\subsubsection{Subject Found and Confirmed}
\begin{enumerate}
    \item Agent A detects the \Gls{subject} in sector S.
    \item A broadcasts an Origin rumour: ''Subject found at location L.''
    \item Neighbours go to L to confirm Subject existence at L.
    \item Neighbours broadcast Source rumours based on their own observations.
    \item Agents that have received three Source rumours interpret the event through internal voting.
    \item A decides from Source rumours that the Subject is confirmed or that A should move on to other sectors.
    \item If Subject is confirmed found through internal voting, all agents eventually shift to return state.
\end{enumerate}

\subsubsection{Agent Returns for Battery Change}
\begin{enumerate}
    \item \acrshort{fmu} on Agent A reports low battery.
    \item A releases its sectors and broadcasts an Origin rumour: ''Returning for battery change''.
    \item Neighbours interpret the Origin rumour and broadcast Source rumours: ''A no longer member of swarm''.
    \item SwarmInterface displays to the \gls{searchmana}: ''Agent A returning – prepare battery replacement.''
    \item Agents that have received three Source rumours interpret the event through internal voting and update their AgentInfo and SectorInfo lists.
    \item A comes back and requests membership.
\end{enumerate}
