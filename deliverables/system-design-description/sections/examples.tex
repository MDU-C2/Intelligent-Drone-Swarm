\subsection{Dynamic Deployment}
The protocol module supports dynamic and staggered agent deployment to support real-world scenarios that involve battery cycles, repairs, and communication degradation.

\noindent Dynamic deployment means that agents may:
\begin{itemize}
    \item Launch at different times.
    \item Join the swarm after the initial deployment.
    \item Disconnect temporarily due to communication loss.
    \item Rejoin after recovery.
\end{itemize}
A \textbf{returning} agent A must perform a registration procedure:
\begin{enumerate}
    \item Announce its presence through an Origin rumour to agents that were considered neighbours by A.
    \item Receive Source rumours from those neighbours concerning if A has received membership.
    \item Integrate into neighbourhood structures.
    \item Receive information from neighbours to update its AgentInfo and SectorInfo lists.
    \item Request sector allocation.
    \item Synchronise with the swarm state.
\end{enumerate}
A \textbf{new} agent B must perform a registration procedure:
\begin{enumerate}
    \item Before deployment, receive information from the SwarmInterface to populate its AgentInfo and SectorInfo lists, and receive information on possible neighbours.
    \item Announce its presence through an Origin rumour to agents the SwarmInterface assigned as possible neighbours.
    \item Receive Source rumours from those possible neighbours concerning if B has received membership.
    \item Integrate into neighbourhood structures.
    \item Receive information from neighbours to update its AgentInfo and SectorInfo lists.
    \item Request sector allocation.
    \item Synchronise with the swarm state.
\end{enumerate}



\subsection{Operational Scenarios}
The following scenarios illustrate how the layers work together.

\subsubsection{Agent Health Degrades}
\begin{enumerate}
    \item FMU on Agent A reports degraded health (e.g. sensor fault).
    \item A releases its \glspl{sector} and broadcasts an Origin: ''I am degraded (bad\_sensor)''.
    \item Neighbours interpret the Origin rumour and broadcast Source rumours about the degraded state.
    \item Agents that have received three Source rumours interpret the event through internal voting.
    \item Agents broadcast Opinion rumours; A receives enough Opinions and decides to switch to a secondary role.
    \item A broadcasts a new Origin announcing its decision.
    \item Its sectors re-enter the \gls{secmarket}; healthy agents bid and take over.
    \item A moves to a secondary role, such as a communication relay.
\end{enumerate}

\subsubsection{Sector Completed}
\begin{enumerate}
    \item Agent A finishes searching sector S.
    \item A creates an Origin rumour: ''I (A) have completed sector S.''
    \item Neighbours interpret the Origin rumour and broadcast Source rumours confirming that S is completed.
    \item Agents that have received three Source rumours interpret the event through internal voting.
    \item Agents update their SectorInfo.
    \item The budget of A increases by the cost of S, and A bids for a new sector.
\end{enumerate}

\subsubsection{Subject Found and Confirmed}
\begin{enumerate}
    \item Agent A detects the \Gls{subject} in sector S.
    \item A broadcasts an Origin rumour: ''Subject found at location L.''
    \item Neighbours go to L to confirm Subject existence at L.
    \item Neighbours broadcast Source rumours based on their own observations.
    \item Agents that have received three Source rumours interpret the event through internal voting.
    \item A decides from Source rumours that the Subject is confirmed or that A should move on to other sectors.
    \item If Subject is confirmed found through internal voting, all agents eventually shift to return state.
\end{enumerate}

\subsubsection{Agent Returns for Battery Change}
\begin{enumerate}
    \item \acrshort{fmu} on Agent A reports low battery.
    \item A releases its sectors and broadcasts an Origin rumour: ''Returning for battery change''.
    \item Neighbours interpret the Origin rumour and broadcast Source rumours: ''A no longer member of swarm''.
    \item SwarmInterface displays to the \gls{searchmana}: ''Agent A returning – prepare battery replacement.''
    \item Agents that have received three Source rumours interpret the event through internal voting and update their AgentInfo and SectorInfo lists.
    \item A comes back and requests membership.
\end{enumerate}




\subsection{Example Rumour Cycles}
This section gives brief examples of how the Rumour Mill works in typical situations.

These scenarios show how the Rumour Mill links local observations to swarm-wide, consistent decisions without a central controller.

\subsubsection{Degraded Health Event}
\begin{enumerate}
    \item \acrshort{fmu} on agent A detects bad\_sensor.
    \item A releases its sectors and broadcasts an Origin: ''I (A) am degraded: bad\_sensor.''
    \item Neighbours of A interpret it and broadcast Sources confirming the degraded state.
    \item Any agent (including neighbours of A) that receives three Sources updates A's sectors to unassigned, and creates an Opinion rumour (e.g. ''A should release its sectors and become relay'').
    \item A collects Opinion rumours; Once Opinion rumours from $\geq \alpha$ fraction of agents have arrived, A internally decides to switch to secondary\_task (e.g. relay).
    \item A broadcasts a decision-Origin announcing its new role.
    \item The \gls{secmarket} (\Cref{sec:sector-alloc}) reassigns A’s sectors to other agents through bidding.
\end{enumerate}


\subsubsection{Subject Found Event}
\begin{enumerate}
    \item Agent A detects the \gls{subject} at location L and broadcasts an Origin: ''Subject found at L.''
    \item Neighbours receive the Origin rumour and go to L to confirm that the subject is at L.
    \item Neighbours broadcast Source rumours based on their findings (e.g. ''Subject not found at L'' or ''Subject found at L'').
    \item Once any agent has three Sources (including originator and neighbours), it interprets the event by either continuing its current task (''Subject not found at L'') or updating their state machines to enter roles appropriate for mission completion (usually return to base).
\end{enumerate}



\subsection{Ownership Changes via Rumour Mill}
\label{subsec:ownership}
All changes in \gls{sector} ownership (initial purchases, reassignments, and releases) are confirmed through the Rumour Mill to avoid conflicting views. A typical sequence for a successful purchase of sector S by agent A is:
%A typical sector purchase by agent A for sector S proceeds as:
\begin{enumerate}
    \item Origin (bid)
    \begin{itemize}
        \item A broadcasts an Origin: ''I (A) want to purchase sector S.''
        \item This Origin uses a specific msgID that indicates a sector bid.
    \end{itemize}
    \item Sources (validation), i.e., 
    \begin{itemize}
        \item Neighbours that receive the Origin evaluate the bid based on their local information:
        \begin{itemize}
            \item[$\circ$] Current owner of S (if any).
            %\item A's budget (as known in AgentInfo).
            \item[$\circ$] Any conflicting bids they have observed.
        \end{itemize}
        \item Each neighbour broadcasts a Source saying, for example, ''Bid from A on S is valid.''
    \end{itemize}
    \item Opinion (summary)
    \begin{itemize}
        \item Once an agent has three Source rumours for this Origin rumour, it broadcasts an Opinion summarising the outcome (e.g. “A should own S” or “A should not own S”).
    \end{itemize}
    \item Decision at A
    \begin{itemize}
        \item A collects Opinions until it has received Opinions from at least $\alpha$ fraction of agents (e.g. 70\% in this example).
        \item A makes a local decision: bid success or failure.
    \end{itemize}
    \item Decision-Origin (announcement)
    \begin{itemize}
        \item A broadcasts a new Origin announcing the decision, where success → ownerID(S) := A and A’s budget is reduced, and failure → no change to ownerID(S).
    \end{itemize}
\end{enumerate}
All agents update their SectorInfo and budgets accordingly. Because the final ownership decision is always announced via a decision-Origin, all agents converge on a consistent view of sector ownership.




\subsection{Reallocation Due to Degradation or Failure}
When an agent becomes degraded or fails the \gls{secmarket} must quickly reassign its sectors to maintain search coverage.

\subsubsection{Sector Release from Degraded Agents}
When an agent becomes degraded (e.g. bad\_sensor state):
\begin{enumerate}
    \item The agent’s \acrshort{fmu} triggers a health event.
    \item The agent releases its sector and starts a Rumour Mill event with an Origin describing its degraded state.
    \item Neighbours of the agent interpret it and broadcast Sources confirming the degraded state
    \item Any agent (including neighbours) that receives three Sources updates A’s sectors to unassigned, and creates an Opinion rumour (e.g. ”A should become relay”).
    \item The agent collects Opinion rumours; Once Opinion rumours from$\geq \alpha$ fraction of agents have arrived, the agent internally decides to switch to secondary\_task (e.g. relay).
    \item The agent broadcasts a decision-Origin announcing its new role.
    \item The Sector Allocation Market (\Cref{sec:sector-alloc}) reassigns A’s sectors to other agents through bidding.
\end{enumerate}

\subsubsection{Sector Release from Failed or Missing Agents}
\label{subsubsec:sector-release-from-failed}
If an agent A fails or disappears (e.g. pulse timeout):
\begin{enumerate}
    \item Neighbours detect missing pulse and create Sources: “Agent A is no longer a member of swarm.”
    \item All agents, after receiving three sources, vote locally on whether Agent A should be considered a member or not.
    \item If considered not a member, all agent A's sectors are treated as released.
    \item SectorInfo entries for sectors formerly owned by A are updated to ownerID = 0.
    %\item Sources and Opinions confirm that X is missing or failed.
    %\item Once the decision is made, X is removed from AgentInfo lists, and its sectors are treated as released.
    %\item SectorInfo entries for sectors formerly owned by X are updated to ownerID = 0.
\end{enumerate}
Again, these sectors become candidates for new bids by healthy agents.

\subsubsection{Re-bidding on Released Sectors}
%Healthy agents monitor SectorInfo:
Agents monitor SectorInfo:
\begin{itemize}
    \item When a sector’s ownerID becomes 0, it is considered available.
    \item Agents with sufficient budget and good cost for that sector may initiate new bids using the process in \Cref{subsec:ownership}.
    \item Over time, released sectors are redistributed to agents best able to search them.
\end{itemize}
This provides automatic replanning: the swarm reorganises its workload in response to agent degradation and failure without central control.


\subsection{From Sector Ownership to Tasks}
For a healthy agent, its primary tasks are derived from the set of sectors it owns:
\begin{itemize}
    \item Each owned sector S corresponds to a search task: fly to S, search S, and report completion.
    \item Tasks can be scheduled internally in some order (e.g. nearest-first or probability-weighted order).
    \item When a sector is fully searched, the agent:
    \begin{enumerate}
        \item Broadcasts an Origin rumour indicating ''sector S completed'',
        \item participates in the Rumour Mill to confirm completion,
        \item updates its local SectorInfo (status = completed),
        \item receives the budget refund for S when the completion decision is confirmed (\Cref{subsec:agent-budget}).
    \end{enumerate}
\end{itemize}
Thus, sector ownership drives the agent’s task queue. The Task \& Role layer connects: SectorInfo to a set of search tasks, health and decisions to whether the agent should continue searching, release sectors, or take on a new role.

After completing or releasing sectors, the agent can use the Sector Allocation Market (\Cref{sec:sector-alloc}) to bid for new sectors, creating new tasks.