\section{System Context}
\label{sec:system-context}
This section describes the operational context, external actors, system boundaries, assumptions, and constraints of the IRDS system.

\subsection{Operational Environment}
The IRDS system is intended for \acrshort{sar} missions in which a swarm of \acrshortpl{uav} is deployed to locate a missing \gls{subject} within a geographical area. Typical characteristics of this environment include:
\begin{itemize}
    \item Uncertain and dynamic environments, including forests, mountain regions, or water.
    \item Limited or unreliable communication caused by obstacles, distance, or environmental interference.
    \item Decentralisation requirements, since persistent connection to a ground station or ''mother drone'' cannot be guaranteed.
    \item Safety and time constraints, where prompt detection of the health of degraded agents is essential to maintain the integrity of the mission.
\end{itemize}
The above conditions motivate a decentralised, fail-operational swarm: The system must be able to continue a mission even if some \glspl{agent} become degraded, fail, or temporarily lose communication.


\subsection{Actors \& External Entities}
The protocol module interacts with several external entities, and by understanding these actors, it clarifies the boundaries between the protocol module, the UAV hardware, and operational human control.

\vspace{0.2cm}

\subsubsection{Search Manager (Human Operator)}
The \gls{searchmana} is responsible for the strategic oversight of the mission. 

Responsibilities include:
\begin{itemize}
    \item Defining and adjusting the \gls{searcharea}.
    \item Marking \glspl{hotregion} on the Search Area.
    \item Initialisation of swarm deployment.
    \item Issuing high-level operational commands (start search, pause, adjust Search Area, abort mission).
    \item Receive status updates through the SwarmInterface.
    \item Preparing interventions such as battery replacements when an agent returns home.
\end{itemize}
The Search Manager does not directly coordinate agent-level behaviour. Instead, all commands are processed through the protocol at swarm level.

\vspace{0.2cm}

\subsubsection{SwarmInterface (External System)}
\label{subsubsec:swarmint}
The SwarmInterface serves as the communication layer between the Search Manager and the swarm. The interface may be implemented as software running on a laptop, tablet, or portable ground station.

Responsibilities include:
\begin{itemize}
    \item Allowing a \gls{searchmana} to draw Hot Regions on a map.
    \item Dividing a Search Area into rectangular \glspl{sector}.
    \item Assigning probability values to sectors based on marked Hot Regions.
    \item Sending high-level commands to the swarm (e.g. mission start, pause, abort mission).
    \item Receiving and displaying aggregated swarm status such as:
    \begin{itemize}
        \item[$\circ$] Degraded health events,
        \item[$\circ$] agents returning for battery replacement,
        \item[$\circ$] \gls{subject} found,
        \item[$\circ$] emergency landings,
        \item[$\circ$] membership status of agents.
    \end{itemize}
    \item Ensure that messages follow the protocol module’s formatting rules, such as message IDs and data structure.
\end{itemize}
The SwarmInterface is included as a conceptual component in this iteration and the IRDS protocol module is designed to integrate with it. The SwarmInterface is intended to be developed in future work.

%\vspace{0.2cm}

\subsubsection{Agents (UAVs)}
\Glspl{agent} are the autonomous \acrshortpl{uav} that execute the protocol module. 

Responsibilities include:
\begin{itemize}
    \item Participate in consensus through the \gls{rumourmill}.
    \item Perform health-dependent behaviour.
    \item Maintain knowledge of its neighbourhood.
    \item Broadcast and receive pulses (heartbeat messages).
    \item Perform tasks or secondary roles according to their state.
    \item Maintain an internal AgentInfo list describing known agents.
    \item Maintain a SectorInfo list describing known sectors.
    \item Form neighbourhoods with geographically closest agents.
    \item Report its health status to the swarm.
    \item Respond to \gls{searchmana} commands through the protocol module.
\end{itemize}
Agents are individually responsible for maintaining internal consistency and are cooperatively responsible for maintaining swarm-wide agreement. Agents can join, rejoin, or leave the swarm at any time.

\vspace{0.2cm}

\subsubsection{Fault Management Unit}
Each \gls{agent} contains a \acrfull{fmu} that monitors the health status of hardware and software. 

Responsibilities include:
\begin{itemize}
    \item Detect degraded health states.
    \item Publish health information to the agent.
    \item Trigger the creation of Origin rumours.
    \item Initiate fault isolation actions.
\end{itemize}
The protocol module relies on FMU health status outputs and does not modify the FMU itself.

\vspace{0.2cm}

\subsubsection{Environment}
The environment influences communication, agent behaviour, and mission performance. 
Relevant environmental factors include:
\begin{itemize}
    \item Radio communication range and interference.
    \item Physical obstacles (terrain, forest canopy, urban structures).
    \item GPS conditions.
    \item Wind, rain, or weather effects on flight stability.
\end{itemize}
The protocol module assumes that while communication may be unreliable, rumour propagation is eventually possible.


\subsection{System Boundaries}
The protocol module defines behaviour at the swarm coordination level. The following responsibilities lie within the scope of the protocol module:
\begin{itemize}
    \item Consensus formation (rumour lifecycle) for event interpretation.
    \item \Gls{sector} allocation and bidding.
    \item Broadcasting health status
    \item Task allocation and reallocation
    \item Neighbourhood formation and recalculation.
    \item Transitions in agent behaviour state.
    \item Propagation of \gls{searchmana} commands.
    \item Track dynamic information about agents and sectors.
    \item Pulse monitoring.
\end{itemize}
\newpage \noindent The following responsibilities fall outside the scope of the protocol:
\begin{itemize}
    \item Low-level flight control (managed by the FCU).
    \item Navigation and GPS fusion.
    \item Obstacle detection and avoidance (ODU / CAU).
    \item Motor control.
    \item Physical take-off, landing, and stability control.
    \item Fault detection logic inside the FMU.
    \item Cryptographic security or authentication mechanisms.
\end{itemize}
%This separation ensures that the protocol module remains lightweight and platform-independent.

\subsection{Assumptions}
The protocol module design relies on the following assumptions:
\begin{itemize}
    \item Swarm size: 4–255 agents
    \begin{itemize}
        \item[$\circ$] A minimum of four agents is required to tolerate one Byzantine fault per neighbourhood.
        \item[$\circ$] The maximum is capped by 8-bit (one byte) AgentID space.
    \end{itemize}
    \item Each agent knows its own ID and can uniquely identify others.
    \item AgentID = 0 reserved for \gls{searchmana}.
    \item Neighbourhood definition:
    \begin{itemize}
        \item[$\circ$] Each agent forms a neighbourhood using the three geographically closest agents, determined by initial \gls{sector} assignment.
        \item[$\circ$] Neighbourhoods overlap.
    \end{itemize}
    \item Neighbourhood recalculation:
    \begin{itemize}
        \item[$\circ$] If an agent fails, leaves, or is kicked out of the swarm, all affected neighbourhoods must be recomputed to restore a 1 + 3 structure.
    \end{itemize}
    \item Communication:
    \begin{itemize}
        \item[$\circ$] Agents exchange messages over unreliable wireless networks.
        \item[$\circ$] Message delays and drops are possible, but connectivity is eventually restored.
    \end{itemize}
    \item \gls{searcharea}:
    \begin{itemize}
        \item[$\circ$] The overall Search Area is defined by the Search Manager via the SwarmInterface.
        \item[$\circ$] The SwarmInterface divides the Search Area into \glspl{sector}.
        \item[$\circ$] Sectors are rectangular (defined by NE/NW/SE/SW points).
        \item[$\circ$] The SwarmInterface assigns a value to each sector based on the Hot Regions marked by the Search Manager.
        \item[$\circ$] That value is used as part of the sector cost in the \gls{secmarket}.
    \end{itemize}
    \item FMU health statuses are trustworthy, meaning that the FMU is assumed to operate correctly and FMU faults are not considered.
    \item Agent dynamics:
    \begin{itemize}
        \item[$\circ$] Agents are airborne during operation except during emergency landings or returns.
        \item[$\circ$] Agents may join, rejoin, or leave the swarm at any time.
    \end{itemize}
    \item Security:
    \begin{itemize}
        \item[$\circ$] No encryption, authentication, or integrity protection is implemented in the current design.
        \item[$\circ$] All agents are assumed to be authorised participants.
        \item[$\circ$] Security is treated as future work.
    \end{itemize}
\end{itemize}

\newpage
\subsection{Constraints}
The following constraints affect the design and behaviour of the protocol module:
\begin{itemize}
    \item Energy constraints:
    \begin{itemize}
        \item[$\circ$] Battery level influences the budget for sector bidding and mobility.
    \end{itemize}
    \item Computational limitations:
    \begin{itemize}
        \item[$\circ$] Agents may run on microcontrollers with limited memory and CPU.
    \end{itemize}
    \item Bandwidth limitations:
    \begin{itemize}
        \item[$\circ$] Message sizes must remain small; rumour structures must be lightweight.
    \end{itemize}
    \item Latency:
    \begin{itemize}
        \item[$\circ$] Rumour consensus must function under delayed or intermittent communication.
    \end{itemize}
    \item Real-time constraints:
    \begin{itemize}
        \item[$\circ$] The protocol module must not delay or interfere with safety-critical flight control routines.
    \end{itemize}
    \item Regulatory constraints:
    \begin{itemize}
        \item[$\circ$] UAV operations must comply with applicable aviation rules and safety procedures.
    \end{itemize}
\end{itemize}

\subsection{Dynamic Deployment}
The protocol module supports dynamic and staggered agent deployment to support real-world scenarios that involve battery cycles, repairs, and communication degradation.

\noindent Dynamic deployment means that agents may:
\begin{itemize}
    \item Launch at different times.
    \item Join the swarm after the initial deployment.
    \item Disconnect temporarily due to communication loss.
    \item Rejoin after recovery.
\end{itemize}
A \textbf{returning} agent A must perform a registration procedure:
\begin{enumerate}
    \item Announce its presence through an Origin rumour to agents that were considered neighbours by A.
    \item Receive Source rumours from those neighbours concerning if A has received membership.
    \item Integrate into neighbourhood structures.
    \item Receive information from neighbours to update its AgentInfo and SectorInfo lists.
    \item Request sector allocation.
    \item Synchronise with the swarm state.
\end{enumerate}
A \textbf{new} agent B must perform a registration procedure:
\begin{enumerate}
    \item Before deployment, receive information from the SwarmInterface to populate its AgentInfo and SectorInfo lists, and receive information on possible neighbours.
    \item Announce its presence through an Origin rumour to agents the SwarmInterface assigned as possible neighbours.
    \item Receive Source rumours from those possible neighbours concerning if B has received membership.
    \item Integrate into neighbourhood structures.
    \item Receive information from neighbours to update its AgentInfo and SectorInfo lists.
    \item Request sector allocation.
    \item Synchronise with the swarm state.
\end{enumerate}

\subsection{System Context Diagram}
\label{subsec:context-diagram}
\begin{figure}[H]
    \centering
    \includegraphics[width=\linewidth]{figures/System-Design/SD-01_BDD - Protocol Context 1.0.png}
    \caption{\textit{(SD-01)} \acrfull{bdd} of which systems the protocol is related to. One swarm contains four to 255 agents, and one \gls{agent} contains one \acrshort{fmu}. \acrshortpl{uav} inherit from agent, meaning that \acrshortpl{uav} must contain \acrshortpl{fmu}. Each agent executes the protocol. The protocol carries out a set of behaviours. \glspl{searchmana} use a Swarm Interface to interact with a swarm.}
    \label{fig:bdd-context}
\end{figure}