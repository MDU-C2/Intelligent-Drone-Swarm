\section{Agent Task \& Role Allocation (Behaviour Layer)}
\label{sec:task-role}
%While the \gls{secmarket} decides which agent owns which \glspl{sector}, the Agent Task \& Role Allocation layer decides what each agent actually does at any given time. It maps sector ownership, health status, and external commands into concrete behaviours such as search, relay, return, or fail safely.

The Agent Task \& Role Allocation layer determines what each agent actually does at any given time, based on which sectors it currently owns (from the Sector Allocation Market, \Cref{sec:sector-alloc}), its health status (from the FMU), and high-level commands or mission status (via the \gls{rumourmill} and SwarmInterface).

This layer is implemented as a state machine inside each agent. While the Sector Allocation Market answers ''who owns which sectors?'', the Task \& Role layer answers ''what should this agent be doing right now?''.

\subsection{From Sector Ownership to Tasks}
For a healthy agent, its primary tasks are derived from the set of sectors it owns:
\begin{itemize}
    \item Each owned sector S corresponds to a search task: fly to S, search S, and report completion.
    \item Tasks can be scheduled internally in some order (e.g. nearest-first or probability-weighted order).
    \item When a sector is fully searched, the agent:
    \begin{enumerate}
        \item Broadcasts an Origin rumour indicating ''sector S completed'',
        \item participates in the Rumour Mill to confirm completion,
        \item updates its local SectorInfo (status = completed),
        \item receives the budget refund for S when the completion decision is confirmed (\Cref{subsec:agent-budget}).
    \end{enumerate}
\end{itemize}
Thus, sector ownership drives the agent’s task queue. The Task \& Role layer connects: SectorInfo to a set of search tasks, health and decisions to whether the agent should continue searching, release sectors, or take on a new role.

After completing or releasing sectors, the agent can use the Sector Allocation Market (\Cref{sec:sector-alloc}) to bid for new sectors, creating new tasks.

\subsection{Roles \& Task Types}
Agents may operate in different roles depending on their health status, \gls{sector} ownership, and mission context. Roles are implemented as combinations of state machine states and allowed behaviours.
%Agents can operate in different roles depending on their health state and mission context.

\vspace{0.2cm}

\subsubsection{Primary Roles}
\textbf{Searcher}
\begin{itemize}
    \item The agent’s main role is to search assigned sectors.
    \item It selects a sector from its task queue, navigates there, and executes the search.
    \item It reports sector completion via the Rumour Mill and moves on to the next task.
\end{itemize}
\noindent \textbf{Subject confirmer}
\begin{itemize}
    \item When an agent finds the \gls{subject}, it raises a ''Subject found'' event (Origin).
    \item It participates in the subsequent Rumour Mill cycle to confirm or reject the detection.
    \item[] (Receiving neighbours go to the location and confirm the subject is there before broadcasting sources about subject found.)
    \item Once the Subject is confirmed found, its role typically transitions to returning to base.
\end{itemize}

\vspace{0.2cm}

\subsubsection{Secondary Roles (for Degraded Agents)}
\textbf{Communication relay}
\begin{itemize}
    \item The agent positions itself to improve communication coverage between other agents and/or the SwarmInterface.
    \item It may hover in a location that bridges otherwise disconnected parts of the swarm.
    \item It continues to send and receive pulses and rumours, but does not perform primary search.
\end{itemize}
\noindent \textbf{Secondary searcher}
\begin{itemize}
    \item If the agent is partially degraded but still somewhat usable, it may be assigned less critical search tasks (e.g. low-probability sectors).
    \item The policies for when to allow this are design choices and can be tuned in future work.
\end{itemize}

\subsubsection{Terminal or Restricted Roles}
\textbf{Return}
\begin{itemize}
    \item The agent returns to base, typically due to low battery, mission completion, or explicit command.
    \item Before returning, it releases its sectors via the \gls{secmarket}, so that other agents can take over.
\end{itemize}
\textbf{Failed}
\begin{itemize}
    \item The agent has suffered a critical failure or emergency landing and is effectively out of service.
    \item It is removed as a member of the swarm after a \gls{rumourmill} decision based on missing pulses or explicit failure events.
    \item Its sectors are released and reallocated to other agents.
\end{itemize}
These roles govern which actions are allowed for the agent and how it contributes to the swarm. The exact mapping from health states to roles is configurable, but the protocol provides the mechanisms to implement these mappings in a decentralised way.

\subsection{Role Selection Logic}
Role selection is driven by three main inputs:
\begin{itemize}
    \item Health status from the \acrshort{fmu} (e.g. healthy, degraded, bad\_sensor, failed),
    \item Sector ownership (how many sectors the agent owns and their status),
    \item External or global decisions (e.g. \gls{searchmana} commands, \gls{subject} confirmed found, swarm-wide abort).
\end{itemize}
The IRDS protocol module provides the mechanisms (Rumour Mill and Sector Allocation Market) to coordinate these inputs; specific role policies can be tuned, but typical examples include:

\begin{itemize}
    \item Bad Sensor $\rightarrow$ Secondary Role + Sector Release
    \begin{enumerate}
        \item FMU detects a bad sensor condition.
        \item Agent releases its sectors and raises an Origin: ''I have bad\_sensor.''
        \item Neighbours broadcast Sources confirming degradation.
        \item Opinions form; once the consensus threshold is met, the agent decides to transition to a secondary role (for example, communication relay).
        \item A decision-Origin is broadcast to announce this outcome.
        \item[] Result: The degraded agent stops performing primary search but remains useful to the swarm.
    \end{enumerate}
    \item Low Battery $\rightarrow$ Return + Sector Release
    \begin{enumerate}
        \item FMU detects that battery is below a safe threshold.
        \item Agent starts a Rumour Mill event indicating low battery and its intention to return.
        \item Sources confirm that returning is appropriate.
        \item The agent releases its sectors and transitions to the return role.
        \item The SwarmInterface can display this event so the Search Manager can prepare a battery change.
        \item[] Result: The agent returns safely without leaving unfinished sectors ''dangling''.
    \end{enumerate}
    \item Subject Found $\rightarrow$ Swarm-Wide Role Change
    \begin{enumerate}
        \item An agent finds the Subject and broadcasts an Origin: ''Subject found at location L.''
        \item[] Neighbours go to the location and confirm the subject is there before broadcasting sources about subject found.
        \item The Rumour Mill confirms or rejects this event.
        %\item Once confirmed, the confirming agent broadcasts a decision-Origin: ''Subject found confirmed.''
        \item All agents receive this decision and transition from search or task roles to return.
        \item[] Result: The entire swarm reacts consistently when the mission objective has been achieved.
    \end{enumerate}
\end{itemize}
These examples show that role decisions are local (each agent runs its own state machine) but are driven by global information provided by the Rumour Mill and Sector Allocation Market.

Future work may extend this role selection logic by including neighbourhood suggestions, for example, to choose optimal relay positions for degraded agents.

\subsection{Agent State Machine}
\label{subsec:state-machine}
The Agent Task \& Role Allocation layer is implemented as a state machine. Each agent transitions between states based on health events, \gls{sector} allocation, and \gls{rumourmill} decisions.

Each agent has the following key states:
\begin{itemize}
    \item start
    \begin{itemize}
        \item[$\circ$] Agent inactive
        \item[$\circ$] Transitions to task upon initialisation or joining
    \end{itemize}
    \item task
    \begin{itemize}
        \item[$\circ$] Receives assigned sectors
        \item[$\circ$] Participates in bidding
        \item[$\circ$] Waits for rumour confirmation
    \end{itemize}
    \item[] task transitions to:
    \begin{itemize}
        \item[$\circ$] search when a sector becomes assigned
        \item[$\circ$] secondary\_task if degraded
        \item[$\circ$] return if failed or commanded by \gls{searchmana}
        \item[$\circ$] failed if agent suffers critical failure
    \end{itemize}
    \item search
    \begin{itemize}
        \item[$\circ$] Actively searches assigned sector(s)
        \item[$\circ$] Broadcasts completion via Origin
        \item[$\circ$] Updates position, health, and pulses
    \end{itemize}
    \item[] search transitions to:
    \begin{itemize}
        \item[$\circ$] task when search is completed
        \item[$\circ$] found if \gls{subject} is detected
        \item[$\circ$] bad\_sensor if FMU detects sensor degradation
        \item[$\circ$] return if commanded
    \end{itemize}
    \item found
    \begin{itemize}
        \item[$\circ$] Occurs when the agent detects the Subject
        \item[$\circ$] Broadcasts Origin: ''Subject found''
    \end{itemize}
    \item[] found transitions to:
    \begin{itemize}
        \item[$\circ$] confirmed when receiving sources from neighbours
    \end{itemize}
    \item confirmed
    \begin{itemize}
        \item[$\circ$] All agents move to return state
        \item[$\circ$] Mission ends or transitions to rescue operations
    \end{itemize}
    \item return
    \begin{itemize}
        \item[$\circ$] Agent returns to base
        \item[$\circ$] Human operator may prepare battery replacement
        \item[$\circ$] Full swarm may return (e.g., mission abort or confirmed subject found)
    \end{itemize}
    \item bad\_sensor
    \begin{itemize}
        \item[$\circ$] Triggered by FMU health output
        \item[$\circ$] Agent broadcasts degraded health
        \item[$\circ$] Drops sectors via rumour process
    \end{itemize}
    \item[] bad\_sensor transitions to:
    \begin{itemize}
        \item[$\circ$] secondary\_task
        \item[$\circ$] or failed
    \end{itemize}
    \item secondary\_task
    \item[] Agent performs a non-critical role, such as:
    \begin{itemize}
        \item[$\circ$] Communication relay.
        \item[$\circ$] Secondary search.
    \end{itemize}
    \item[] secondary\_search transitions to:
    \begin{itemize}
        \item[$\circ$] return if battery low
        \item[$\circ$] failed if further degradation occurs
    \end{itemize}
    \item failed
    \begin{itemize}
        \item[$\circ$] Critical degradation or emergency landing
        \item[$\circ$] Agent informs neighbours (if possible)
        \item[$\circ$] Leaves swarm
        \item[$\circ$] Neighbourhoods recalculate
        \item[$\circ$] Sectors re-enter auction pool
    \end{itemize}
\end{itemize}
Transitions between these states are triggered by:
\begin{itemize}
    \item \acrshort{fmu} health outputs (e.g. bad\_sensor, low battery, failure),
    \item \gls{rumourmill} decisions (e.g. \gls{subject} confirmed found, sector completion, global abort),
    \item \gls{secmarket} events (e.g. no more sectors owned, sectors reallocated),
    \item high-level commands via the SwarmInterface.
\end{itemize}

\subsubsection{State Diagram}
\begin{figure}[H]
    \centering
    \includegraphics[width=\linewidth]{figures/System-Design/SD-18_SD - Agent Stateflow 1.0.png}
    \caption{\textit{SD-18} \gls{statedia} of an individual agent.}
    \label{fig:stateflow}
\end{figure}

\begin{figure}[H]
    \centering
    \includegraphics[width=\linewidth]{figures/System-Design/SD-27_SD - Fully Operational 1.0.png}
    \caption{\textit{SD-27} \gls{statedia} of an individual agent's fully operational state.}
    \label{fig:fully-op}
\end{figure}

\begin{figure}[H]
    \centering
    \includegraphics[width=\linewidth]{figures/System-Design/SD-26_SD - Partially Operational 1.0.png}
    \caption{\textit{SD-26} \gls{statedia} of an individual agent's partially operational state.}
    \label{fig:partially-op}
\end{figure}

\subsubsection{Activity model: moving to the next sector}
\color{red}Activity diagram showing the process of an agent moving on to its next sector when it has finished searching a sector. Will include broadcast of "sector X finished"\color{black}

\newpage

\subsection{Swarm-Level Behaviour from Local Roles}
Although each agent runs its own state machine and makes decisions locally, the combination of globally consistent \gls{sector} ownership (\Cref{sec:sector-alloc}), globally consistent event interpretation (\Cref{sec:rumour-mill}), and consistent role selection rules (this section) leads to coherent swarm-level behaviour.

Examples of emergent behaviours include:
\begin{itemize}
    \item Collective Replanning:
    \begin{itemize}
        \item[$\circ$] When one agent degrades or fails, its sectors are released and reallocated.
        \item[$\circ$] Healthy agents take over these sectors by bidding in the \gls{secmarket}.
        \item[$\circ$] The swarm automatically reshapes its search pattern to compensate for the loss.
    \end{itemize}
    \item Coherent Mission Termination:
    \begin{itemize}
        \item[$\circ$] When the \gls{subject} is confirmed found, Source rumours are broadcast.
        \item[$\circ$] All agents update their state machines to transition to return (or other appropriate end-of-mission roles).
        \item[$\circ$] The swarm thus terminates the search phase in a coordinated way without conflicting behaviours.
    \end{itemize}
    \item Self-Healing After Failures
    \begin{itemize}
        \item[$\circ$] Missing pulses cause suspected failures to be raised via the \gls{rumourmill}.
        \item[$\circ$] Once a failure is confirmed, the failed agent’s sectors are released and reallocated.
        \item[$\circ$] Neighbourhoods are recomputed to maintain the 1 + 3 structure where possible.
        \item[$\circ$] Information about the failed agent (e.g. role, sectors, health) is cleaned up in AgentInfo.
    \end{itemize}
\end{itemize}
As a result, the swarm can heal from local failures and continue its mission using only local state machines and broadcast rumour messages. This shows how the system-level properties described in the safety and reliability section are realised through local behaviours.

\subsection{Safety-Oriented Role Restrictions}
Role allocation is also used to enforce safety constraints, for example:
\begin{itemize}
    \item[$\circ$] Agents in bad\_sensor or other degraded states cannot perform primary search of critical \glspl{sector}.
    \item[$\circ$] Agents in failed state must not rejoin the swarm without a proper registration sequence.
    \item[$\circ$] Agents in return should not acquire new sectors until after maintenance and rejoin, if at all.
\end{itemize}
These policy-level restrictions, combined with the state machine, ensure that degraded or failed agents do not perform unsafe or misleading actions. Their impact on the mission is limited to safe behaviours (e.g. relay) or removal from the swarm.

Further safety analysis of these behaviours is provided in \Cref{sec:safety-reliability}.