% !TeX root = ../main.tex
\section{Safety \& Reliability}
\label{sec:safety-reliability}
%The protocol enhances safety at both the individual agent level and the collective swarm level, preventing mission-compromising events such as search gaps, inconsistent behaviour, or cascading failures. This section describes the safety mechanisms, reliability properties, and fault-tolerance strategies embedded into the protocol.
The IRDS protocol module is designed to keep a \acrshort{sar} mission running safely and coherently even when individual agents degrade, fail, or behave unpredictably. This section describes:
\begin{itemize}
    \item The safety objectives of IRDS
    \item The fault model and pulse-based failure detection.
    \item How the agent state machine enforces safe behaviour.
    \item The formal assumptions behind fault tolerance.
    \item How decentralisation contributes to mission-level reliability.
\end{itemize}

\subsection{Safety Objectives}
%The protocol supports the following safety-related objectives to ensure that the swarm operates safely in uncertain, dynamic \acrshort{sar} environments:
The protocol module aims to satisfy the following safety objectives:
\begin{itemize}
    \item Mission continuity: The swarm must continue searching even if individual agents degrade or fail.
    \item Consistent decision-making: All non-faulty agents should eventually agree on the interpretation of critical events and the resulting actions.
    \item Prevention/Avoidance of contradictory actions: Conflicts over \gls{sector} ownership, search direction, or state transitions must be prevented. The swarm should avoid unsafe conflicts such as two agents believing they own the same sector, or an agent both ''returning'' and ''searching'' simultaneously.
    \item Containment of individual faults: A malfunctioning or malicious agent must not compromise the swarm. The influence of a malfunctioning or malicious agent should be localised; the swarm as a whole should remain trustworthy.
    \item Graceful degradation: Agents must shift to safe secondary tasks or return-home behaviour when necessary. Degraded agents should move into safe secondary roles or return home, rather than silently producing bad data.
    \item Operator awareness: Critical events must be communicated to the \gls{searchmana} in a timely manner. Critical events (e.g. agent returning for battery, agent degraded, \gls{subject} found) should be communicated to the Search Manager via the SwarmInterface.
\end{itemize}
These objectives shape the design of the \gls{rumourmill}, \gls{secmarket}, and Task \& Role layers.

\subsection{Fault Model}
The protocol module assumes the following types of faults may occur:
\begin{enumerate}
    \item Crash Faults
    \item[] The agent stops operating or lands unexpectedly (includes emergency landing), and may stop sending messages entirely.
    \item[] Examples: Hardware failure, sudden power loss, motor loss.
    \item Omission Faults
    \item[] The agent intermittently fails to send pulses, or propagate rumour messages.
    \item[] This can be due to radio interference, range limitations, or transient internal issues.
    \item Timing Faults
    \item[] Messages may be delayed or arrive out of order because of variable communication latency and local processing delays.
    \item[] The protocol does not assume strict timing guarantees; it aims for eventual consistency instead of synchronous behaviour.
    \item Byzantine Faults
    \item[] An agent may behave arbitrarily, including sending incorrect, inconsistent, or misleading information.
    \item[] IRDS is designed to tolerate one Byzantine fault per neighbourhood (details in \Cref{sec:formal-fault-tolerance-model}), provided faulty agents are not too clustered.
\end{enumerate}
The protocol is designed to tolerate one Byzantine fault per neighbourhood and multiple crash or omission faults swarm-wide.

\subsection{Pulse Safety Mechanism}
Pulses provide a simple, continuous check on agent presence (health status is assumed healthy unless a rumour states otherwise). Pulse messages support safety by enabling:
\begin{itemize}
    \item Agent presence detection.
    \item Neighbourhood integrity checks.
    \item Failure detection.
    \item Position updates for separation distance and collision avoidance buffers.
\end{itemize}
Content of a pulse can be seen in \Cref{subsubsec:pulse-content}.

\vspace{0.2cm}

\subsubsection{Pulse-Based Failure Detection}
If an agent does not receive pulses from any of their neighbours for a configured timeout period, it:
\begin{itemize}
    \item Suspects that the neighbour, agent A, has failed or is unreachable,
    \item creates a Source rumour: ''Agent A no longer member of swarm'',
    \item participates in the \gls{rumourmill} to confirm or reject this Source rumour.
    \item Once the Rumour Mill confirms that Y is missing or failed:
    \begin{itemize}
        \item[$\circ$] Y is set as no longer a member in AgentInfo,
        \item[$\circ$] neighbourhoods are recalculated to exclude Y,
        \item[$\circ$] sectors previously owned by Y are considered released and re-enter the \gls{secmarket} (\Cref{subsubsec:sector-release-from-failed}).
    \end{itemize}
\end{itemize}
Thus, silent failures do not go unnoticed and their impact on search coverage is mitigated.


\subsection{Safety-Driven State Behaviour}
The agent state machine (\Cref{subsec:state-machine}) is designed with safety in mind, where key transitions include:
\begin{itemize}
    \item Health degradation $\rightarrow$ restricted roles:
    \begin{itemize}
        \item bad\_sensor $\rightarrow$ transition from search to secondary\_task, prevents faulty sensors from corrupting search results.
        \item Release of \glspl{sector} via the Sector Allocation Market.
        \item Stop performing primary search tasks.
    \end{itemize}
    \item Critical failure $\rightarrow$ failed:
    \begin{itemize}
        \item Severe \acrshort{fmu} errors or complete communication loss $\rightarrow$ failed.
        \item Removal of the agent from neighbourhoods and SectorInfo after Rumour Mill confirmation.
    \end{itemize}
    \item Low battery $\rightarrow$ return:
    \begin{itemize}
        \item Low-battery event $\rightarrow$ sector release, then return.
        \item Prevents agents from dying silently in the field with sectors assigned.
    \end{itemize}
    \item Subject confirmation $\rightarrow$ return
    \begin{itemize}
        \item Subject confirmed found $\rightarrow$ all agents transition towards return.
    \end{itemize}
\end{itemize}
These transitions ensure that degraded agents limit their impact, failed agents are excluded from coordination, and resources and responsibilities are reallocated to healthy agents.

\subsection{Formal Fault Tolerance Model}
\label{sec:formal-fault-tolerance-model}
This subsection summarises the assumptions and reasoning behind the fault tolerance of the IRDS protocol module with respect to neighbourhoods and consensus.

\vspace{0.2cm}

\subsubsection{Neighbourhood Size and Structure}
Each agent forms a neighbourhood consisting of itself and its three geographically closest agents (based on initial sectors). Thus, each neighbourhood contains four agents in total (1 + 3).
Each agent must be seen as a neighbour by at least three other agents.

Neighbourhoods overlap, meaning that a given agent is the centre of its own neighbourhood, and is one of the three neighbours in three or more other agents’ neighbourhoods.

\subsubsection{Rumour Mechanics within a Neighbourhood}
For a given event:
\begin{itemize}
    \item The originator broadcasts an Origin rumour.
    \item Neighbour agents independently broadcast Source rumours interpreting that Origin rumour.
    \item Any agent that receives three Source rumours for this Origin rumour can interpret the event through voting, and possibly creates an Opinion.
\end{itemize}
Within the originator’s neighbourhood, there are up to three neighbour Source rumours (from its three neighbours), and potentially additional Source rumours from other agents as rumours propagate. To interpret an event and possibly broadcast an Opinion rumour, an agent needs three Source rumours, and those Source rumours may come from neighbours and/or other agents in the swarm.

\vspace{0.2cm}

\subsubsection{Tolerance of One Byzantine Neighbour}
Consider a single neighbourhood of 4 agents (A = originator, plus neighbours B, C, D), and assume at most one of these neighbours is Byzantine:
\begin{itemize}
    \item In the worst case, one neighbour (say B) may send misleading or malicious Source rumours.
    \item The other two neighbours (C and D) are non-faulty and produce correct Source rumours.
    \item Any agent forming an Opinion based on three Source rumours will see at least two honest Source rumours out of the three.
\end{itemize}
Since Opinions are derived from multiple independent Source rumours, a single Byzantine neighbour cannot, by itself, force a consistently incorrect Opinion if the Opinion logic respects the majority of honest Source rumours. Therefore with neighbourhood size 4 (1 + 3), the design can tolerate one Byzantine agent per neighbourhood without corrupting the consensus process, assuming that Opinion formation logic is dominated by honest Source rumours.

\vspace{0.2cm}

\subsubsection{Overlapping Neighbourhoods and System-Level Bounds}
Because neighbourhoods overlap, a single Byzantine agent appears as a neighbour in several neighbourhoods, but each neighbourhood still contains at most 1 Byzantine agent as long as faults are not clustered.

A key constraint is for each neighbourhood N$_i$, the number of Byzantine agents in that neighbourhood must be at most 1:
\begin{equation*}
    \forall N_i : |N_i \cap B| \leq 1
\end{equation*}
where B is the set of Byzantine agents.

If this constraint holds, then each neighbourhood’s Opinions are dominated by honest Source rumours, swarm-wide consensus remains trustworthy, and the overall system remains safe, even with multiple Byzantine agents.

Roughly speaking (and ignoring clustering effects), with N agents and neighbourhoods of size 4, the swarm can tolerate up to on the order of N/4 Byzantine agents \textbf{provided they are spread out} so that no neighbourhood contains more than one Byzantine agent. This is an approximation, not a strict bound, and a precise analysis is left for future work.

\vspace{0.2cm}

\subsubsection{Relationship to Consensus Threshold $\alpha$}
The consensus threshold $\alpha$ (\Cref{subsec:consens-threshold}) determines how many Opinions the originator waits for before making a decision.
\begin{itemize}
    \item Larger $\alpha \rightarrow$ more tolerant to faulty or missing agents, but slower decisions.
    \item Smaller $\alpha \rightarrow$ faster decisions, but less robust if many agents are faulty or disconnected.
\end{itemize}
In this design, $\alpha$ is treated as a configurable parameter (70\% used in examples). The optimal value depends on:
\begin{itemize}
    \item Expected number and distribution of faulty agents.
    \item Communication reliability.
    \item Mission time constraints.
\end{itemize}

\subsection{Prevention of Contradictory Actions}

The combination of the \gls{rumourmill}, the \gls{secmarket}, and state machine prevents several classes of contradictory or unsafe actions:
\begin{itemize}
    \item Double \gls{sector} ownership
    \begin{itemize}
        \item[$\circ$] Sector ownership changes are always confirmed through the Rumour Mill.
        \item[$\circ$] Bids are validated by Source rumours and Opinions before the originator announces a successful purchase.
        \item[$\circ$] All agents update their SectorInfo based on the decision-Origin, avoiding conflicting owners.
    \end{itemize}
    \item Inconsistent health interpretations
    \begin{itemize}
        \item[$\circ$] Health changes (e.g. degraded, failed) are rumour events.
        \item[$\circ$] Agents eventually converge on whether an agent is healthy, degraded, or failed.
        \item[$\circ$] This avoids half the swarm thinking an agent is healthy while others think it has failed.
    \end{itemize}
    \item Conflicting roles
    \begin{itemize}
        \item[$\circ$] The state machine defines allowed transitions (e.g. from search to return, not both at once).
        \item[$\circ$] Rumour Mill decisions (e.g. \gls{subject} confirmed, mission aborted) lead to consistent role changes across the swarm.
    \end{itemize}
\end{itemize}
These mechanisms prevent the most dangerous inconsistencies during mission execution.

\subsection{Eventual Consistency Guarantees}
The protocol module assumes a lossy, broadcast communication channel, i.e., not every agent receives every message. However:
\begin{itemize}
    \item Origins, Sources, Opinions, and decision-Origins are broadcast repeatedly over time,
    \item neighbourhoods overlap, creating multiple paths for information to propagate,
    \item agents maintain rumour state and can act on messages received later.
\end{itemize}
Under these conditions:
\begin{itemize}
    \item all non-faulty agents will eventually receive the decision-Origin for each rumour,
    \item their local AgentInfo and SectorInfo will converge to the same values,
    \item their state machines will react consistently.
\end{itemize}
This eventual consistency is sufficient for SAR missions where real-time, perfectly synchronised decisions are not strictly required, but coherent behaviour over time is.

\subsection{Safety Benefits of Decentralisation}
The decentralised design of the protocol module offers several safety benefits:
\begin{itemize}
    \item No single point of failure
    \begin{itemize}
        \item[$\circ$] No central controller whose failure would stop the mission.
        \item[$\circ$] No special ''leader'' agent that, if compromised, could mislead the entire swarm.
    \end{itemize}
    \item Local containment of faults
    \begin{itemize}
        \item[$\circ$] Faulty behaviour from one agent is limited by neighbourhood cross-checking and the need for multiple Source rumours and Opinions.
        \item[$\circ$] Degraded agents are moved into secondary or terminal roles.
    \end{itemize}
    \item Graceful scaling
    \begin{itemize}
        \item[$\circ$] Adding more agents increases redundancy and potential coverage.
        \item[$\circ$] The same protocol logic applies regardless of swarm size (within design limits).
    \end{itemize}
    \item Resilience to communication loss
    \begin{itemize}
        \item[$\circ$] Because logic is local and rumour-based, temporary disconnects do not permanently break the system; agents can rejoin and resynchronise through the registration procedure.
    \end{itemize}
\end{itemize}


\subsection{Mission-Level Reliability}
At the mission level, the IRDS protocol module improves reliability by:
\begin{itemize}
    \item Maintaining search coverage
    \begin{itemize}
        \item[$\circ$] When agents fail or degrade, their \glspl{sector} are released and reallocated via the \gls{secmarket}.
        \item[$\circ$] The swarm continues searching instead of leaving ''holes'' where failed agents were assigned.
    \end{itemize}
    \item Keeping the operator informed
    \begin{itemize}
        \item[$\circ$] Return requests, degraded states, and mission milestones (e.g. \gls{subject} found) are surfaced to the \gls{searchmana} via the SwarmInterface.
        \item[$\circ$] The operator can prepare interventions such as battery changes or mission replans.
    \end{itemize}
    \item Supporting long-running missions
    \begin{itemize}
        \item[$\circ$] Dynamic join/rejoin allows agents to be swapped in and out (e.g. for battery changes) without restarting the mission.
        \item[$\circ$] The protocol automatically integrates new agents and rebalances the workload.
    \end{itemize}
\end{itemize}
Overall, the combination of \gls{rumourmill} consensus, market-based \gls{sector} allocation, and safety-aware role allocation results in a swarm that can tolerate a range of faults and still perform its SAR mission in a controlled and predictable manner.