\section{Future Work}
\label{sec:future-work}
This section outlines future work to refine and implement the IRDS conceptual architecture, protocol specification, and behavioural model for a decentralised, fail-operational UAV swarm in SAR missions.

\subsection{Protocol Implementation}
The Rumour Mill, Sector Allocation Market, and Task \& Role Allocation are currently specified conceptually and only partially explored in simulation. Implementing them will require:
\begin{itemize}
    \item Translating the protocol logic into an embedded language (e.g. C/C++) suitable for UAV microcontrollers.
    \item Defining concrete message formats and encodings (bit-level layout for msgID, taskID, etc.).
    \item Implementing Origin/Source/Opinion handling and lifecycle on real hardware.
    \item Integrating the rumour lifecycle with periodic broadcast scheduling and pulse transmission.
    \item Implementing local storage for AgentInfo, SectorInfo, etc. within memory constraints.
\end{itemize}
Implementation work should also address:
\begin{itemize}
    \item Handling message drops, duplicates, and stale rumours.
    \item Setting practical limits on the number of active rumours and stored Sources/Opinions.
    \item Memory- and CPU-efficient data structures for onboard use.
\end{itemize}
Fault injection and stress testing in simulation and on hardware will be important to verify real-time behaviour, robustness, and scalability.

\subsection{SwarmInterface Development}
In the current design, the SwarmInterface is a conceptual component. Future work should develop a concrete implementation, including:
\begin{itemize}
    \item A graphical map-based interface for the Search Manager.
    \item Tools to draw Hot Regions and visualise sector boundaries and probabilities.
    \item Controls for sending high-level commands (start, pause, abort, area updates).
    \item Visualisation of swarm status:
    \begin{itemize}
        \item Agent positions and health statuses.
        \item Which sectors are assigned / in progress / completed.
        \item Agents returning for battery change.
        \item Subject found / confirmed events.
    \end{itemize}
    \item Event logging, playback, and debugging tools.
\end{itemize}
Hardware requirements may include a laptop or tablet, and a ground radio (or equivalent) for broadcast communication with the swarm.

Designing this interface with usability in mind will make the system much more practical for real SAR operators.

\subsection{Pathing \& Navigation Integration}
The protocol module currently assumes that agents can move between sectors when requested. Integration with navigation and path planning deserves further development:
\begin{itemize}
    \item Computing efficient routes to visit multiple owned sectors (e.g. nearest-first, probability-weighted).
    \item Integrating with buffer-zone enforcement, and collision avoidance and obstacle detection subsystems.
    \item Dynamic replanning of paths when other agents move, new obstacles are detected, or sectors change ownership.
    \item Ensuring that role changes (e.g. becoming a relay) translate into stable, safe positions in space.
\end{itemize}
This may involve defining a clear interface between the Task \& Role layer and the navigation stack, and experimenting with different path-planning strategies and measuring their impact on mission time and safety.

\subsection{Expanded Secondary Tasks}
Currently, the protocol module defines limited secondary tasks for degraded agents, such as communication relay and secondary search.

Future enhancements may include:
\begin{itemize}
    \item Communication relay positioning:
    \item[] Selecting optimal relay positions based on neighbourhood geometry or sector layout.
    \item Search assistance:
    \item[] Using partially degraded agents to search sectors owned by healthy agents.
\end{itemize}

Additionally, neighbourhoods may vote (via independent Sources) on where a degraded agent should position itself as a communication relay, improving cooperative performance.

Task prioritisation rules may also be expanded to consider mission-critical objectives.

\subsection{Security}
Security is not implemented in the conceptual design, but future work should include an appropriate encryption scheme appropriate for \acrshortpl{uav}.

Suggested starting point:
\begin{itemize}
    \item AES-128 encryption with a shared operation-specific 16-byte key.
    \item Key derived via a \gls{uuid}v4 random generator for each mission.
    \item[] This is intended to ensure that only authorised participants can interpret messages and to provide basic protection against eavesdropping and simple spoofing.
\end{itemize}

Additional future enhancements may include:
\begin{itemize}
    \item Replay protection to prevent old rumours from being re-injected.
    \item Secure registration for join/rejoin procedures.
\end{itemize}

Initial work could focus on low-overhead schemes suitable for microcontrollers and low-bandwidth radio links. More advanced work could study detection of malicious agents that send inconsistent Source rumours or Opinion rumours, and integration of security events into the fault model and Rumour Mill.

\subsection{Scalability \& Optimisation}
The current design targets swarms of roughly 4–255 agents. Future work can investigate:
\begin{itemize}
    \item How performance (message load, decision latency, convergence time) scales with swarm size.
    \item Alternative neighbourhood selection strategies (e.g. dynamic neighbourhood size, non-geographic metrics).
    \item Adaptive rebroadcast strategies to reduce redundant traffic while maintaining robustness.
    \item Optimisation of rumour timeouts and retry strategies.
    \item Techniques for prioritising critical rumours over less important ones.
\end{itemize}
Simulation with larger swarms (e.g. hundreds of agents) can help identify scaling bottlenecks and guide protocol optimisations.

\subsection{Formal Verification \& Validation}
Once implemented (for future dependability or safety case work), the IRDS protocol module should undergo:
\begin{itemize}
    \item Formal modelling using state machines or model-checking frameworks.
    \item Validation against \acrshort{sar} requirements and safety objectives.
    \item Verification of rumour convergence.
    \item Stress-testing under high fault loads.
    \item Analysis of worst-case message overhead.
    \item Verification that no inconsistent states can arise.
    \item Analysis of the impact of different consensus thresholds $\alpha$ under various fault scenarios.
    \item Verification that, under specified assumptions, the protocol tolerates one Byzantine fault per neighbourhood.
\end{itemize}

\subsection{Physical Deployment (Long-Term)}
Eventually, the protocol module may be tested on real UAVs. This will require:
\begin{itemize}
    \item Hardware integration of the protocol module.
    \item Adapting communication models to real radio frequency hardware.
    \item Accounting for UAV delays, vibration, and sensor noise.
    \item Ensuring real-time constraints are met.
    \item Integration with flight controllers.
    \item Safety testing in controlled flight zones.
\end{itemize}

\subsection{Summary}
The IRDS protocol, as described in this document, provides a decentralised Rumour Mill for consensus, a Market for sector allocation, and a Task \& Role state machine for agent behaviour. Future work can extend this foundation by:
\begin{itemize}
    \item Further development of the conceptual designs.
    \item Implementing and optimising the protocol on real hardware.
    \item Building a practical SwarmInterface for SAR operators.
    \item Expansion of secondary tasks.
    \item Integrating advanced path planning and secondary tasks.
    \item Adding cryptographic security.
    \item Scaling and formally verifying the design.
    \item Eventually deploying and testing the system in real-world scenarios.
\end{itemize}
These directions provide a roadmap for future project groups and research efforts to evolve IRDS from a conceptual design into a robust, field-ready swarm system.