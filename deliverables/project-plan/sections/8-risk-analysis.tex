\section{Risk Analysis \& Management}
\label{section:risk}
\normalsize
%This section identifies potential events or conditions that could negatively (or sometimes positively) affect the project and describes how they will be handled.
%Key components:
%- Risk Identification: List possible risks (technical, schedule, resource, external factors).
%- Risk Assessment: Evaluate each risk for likelihood and impact. Often presented in a table.
%- Risk Mitigation / Response: How the team plans to reduce, avoid, transfer, or accept each risk.
%- Monitoring: How risks will be tracked throughout the project.
%Purpose:
%- Minimize surprises and ensure the project can stay on track.
%- Demonstrate proactive planning to stakeholders or evaluators.
%Risk identification, assessment, mitigation strategies
%Monitoring and reporting
\subsection{Identified Risks}
%Source (Dependency / Factor)

\renewcommand{\arraystretch}{1.4}
\begin{table}[H]
    \centering
    \footnotesize
    \begin{tabular}{|c|l|l|c|c|l|}
        \hline
        \textbf{Risk ID} & \textbf{Risk Description} & \textbf{Source} & \textbf{Impact} & \textbf{Likelihood} & \textbf{Mitigation Strategy} \\
        \hline
        R-01 & \makecell{Outdated or incomplete\\ UAV specifications} & \makecell{Reference UAV\\ Specifications} & Medium & Low & \makecell{Validate Carrier H6HL\\ specs early; adjust assumptions\\ if updates occur} \\
        \hline
        R-02 & \makecell{Regulatory changes\\ affecting UAV assumptions} & \makecell{External Regulations\\ \& Standards} & Low & Low & \makecell{Monitor relevant UAV\\ standards; adapt simulations\\ if required} \\
        \hline
        R-03 & \makecell{Reduced progress\\ due to limited time} & Time Constraint & High & Medium & \makecell{Plan internal deadlines with\\ buffer time; avoid scheduling\\ critical activities in parallel} \\
        \hline
        R-04 & \makecell{Sudden illness\\ among team members} & \makecell{Health and Work\\ Environment} & Medium & Low & \makecell{Keep stress levels low,\\ Ensure breaks} \\
        \hline
        R-05 & \makecell{Delayed ''Read Task\\ Description'' (TP-01)} & Time Estimation & High & Low & Adhere to project schedule \\
        \hline
        R-06 & \makecell{Delayed ''Read\\ Previous Work'' (TP-02)} & Time Estimation & High & Low & Adhere to project schedule \\
        \hline
        R-07 & \makecell{Delayed ''Familiarize\\ with SORA'' (TP-03)} & Time Estimation & High & Low & Adhere to project schedule \\
        \hline
        R-08 & \makecell{Delayed ''Write\\ Role Descriptions'' (TP-04)} & Time Estimation & High & Low & Adhere to project schedule \\
        \hline
        R-09 & \makecell{Delayed ''Define\\ Project Goals'' (TP-06)} & Time Estimation & High & Low & Adhere to project schedule \\
        \hline
        R-10 & \makecell{Delayed ''Solution\\ Ideas'' (TP-07)} & Time Estimation & High & Low & Adhere to project schedule \\
        \hline
        R-11 & \makecell{Delayed ''Set up\\ Database'' (TP-08)} & \makecell{Time Estimation,\\ Programmer Availability} & High & Medium & \makecell{Adhere to project schedule,\\ plan work properly, more than\\ one person carries out the task} \\
        \hline
        R-12 & \makecell{Delayed ''Requirements\\ Management Plan'' (TP-14)} & \acrshort{rm} & High & Low & Adhere to project schedule \\
        \hline
        R-13 & \makecell{Delayed ''Enhance\\ Simulation Software'' (TI-01)} & Programmer Availability & High & Low & \makecell{Adhere to project schedule,\\ Start activity early} \\
        \hline
        R-14 & \makecell{Delayed ''Drone\\ Swarm Requirements'' (TI-02)} & \makecell{Time Estimation,\\ Database issue} & High & Medium & \makecell{Adhere to project schedule,\\ more than one person carries\\ out the task} \\
        \hline
        R-15 & \makecell{Delayed ''System\\ Requirements'' (TI-03)} & \makecell{Time Estimation,\\ Database issue} & High & Medium & \makecell{Adhere to project schedule,\\ more than one person carries\\ out the task} \\
        \hline
        R-16 & \makecell{Delayed ''Item\\ Requirements'' (TI-05)} & \makecell{Time Estimation,\\ Database issue} & High & Low & \makecell{Adhere to project schedule,\\ more than one person carries\\ out the task} \\
        \hline
        R-17 & \makecell{Delayed ''Item\\ Design'' (TI-06)} & Time Estimation & High & Low & Adhere to project schedule \\
        \hline
        R-18 & \makecell{Delayed ''Item\\ Integration'' (TI-07)} & Time Estimation & High & Medium & \makecell{Adhere to project schedule,\\ more than one person carries\\ out the task} \\
        \hline
        R-19 & \makecell{Delayed ''System\\ Integration'' (TI-08)} & Time Estimation & High & Medium & Adhere to project schedule \\
        \hline
        R-20 & \makecell{Delayed ''Simulate\\ Solution'' (TI-09)} & \makecell{Time Estimation,\\ Simulation Software} & High & Low & Adhere to project schedule \\
        \hline
        R-21 & \makecell{Delayed ''Drone Swarm\\ Req. Validation'' (TV-03)} & Time Estimation & High & Low & Adhere to project schedule \\
        \hline
        R-22 & \makecell{Delayed ''System Req.\\ Validation'' (TV-04)} & Time Estimation & High & Low & Adhere to project schedule \\
        \hline
        R-23 & \makecell{Delayed ''Item Req.\\ Validation'' (TV-06)} & Time Estimation & High & Low & Adhere to project schedule \\
        \hline
    \end{tabular}
    \caption{Risk Log}
    \label{tab:risk-log}
\end{table}

\begin{table}[H]
    \centering
    \footnotesize
    \begin{tabular}{|c|l|l|c|c|l|}
        \hline
        \textbf{Risk ID} & \textbf{Risk Description} & \textbf{Source} & \textbf{Impact} & \textbf{Likelihood} & \textbf{Mitigation Strategy} \\
        \hline
        R-24 & \makecell{Delayed ''Item\\ Verification'' (TV-08)} & Time Estimation & High & Low & Adhere to project schedule \\
        \hline
        R-25 & \makecell{Delayed ''System\\ Verification'' (TV-09)} & Time Estimation & High & Low & Adhere to project schedule \\
        \hline
        R-26 & \makecell{Delayed ''Drone Swarm\\ Verification'' (TV-10)} & Time Estimation & High & Low & Adhere to project schedule \\
        \hline
        R-27 & \makecell{Delayed\\ ''Presentation'' (TE-02)} & Time Estimation & High & High & \makecell{Adhere to project schedule,\\ work continuously with\\ presentation framework during project} \\
        \hline
        R-28 & \makecell{Delayed \\''Final Report'' (TE-03)} & \makecell{Time Estimation,\\ Delayed Deliverables} & High & High & \makecell{Adhere to project schedule,\\ work continuously with report\\ framework during project} \\
        \hline
        %1 & 2 & 3 & 4 & 5 & 6 \\
        %\hline
    \end{tabular}
    %\caption{Risk Log}
    %\label{tab:risk-log}
    \captionof*{table}{Table \ref{tab:risk-log}: Risk Log}
\end{table}
\renewcommand{\arraystretch}{1.0}

\subsection{Monitoring \& Control}

\begin{itemize}
    \item Risks will be reviewed and identified during weekly scrums, and updates and/or additions will be made in table \ref{tab:risk-log}.
    \item High-impact and medium to high likelihood risks (R-03, R-11, R-14, R-15, R-18, R-19, R-27, and R-28) will be prioritised and closely monitored.
    \item Mitigation actions will be assigned to specific team members to ensure accountability.
    \item Any risk escalation beyond the control of the team will be reported to the project owner and the course coordinator.
    \item Team availability will be monitored regularly to prevent scheduling conflicts. Internal deadlines will be adjusted in advance if reduced working days are expected.
\end{itemize}