\section{Methodology}

\subsection{Tools \& Techniques}
This project relies on a comprehensive set of tools and techniques that support communication, coordination, development, and rigorous documentation:
\renewcommand{\arraystretch}{2.5}
\begin{table}[H]
    \centering
    \begin{tabular}{|l|l|}
        \hline
        \textbf{Tool / Technique} & \textbf{Usage} \\
        \hline
        Project management model & Framework derived from \cite{projektled} \\
        \hline
        Agile Management & \makecell{Overall project management approach; applied via Scrum and\\ \Gls{kanban} in Jira} \\
        \hline
        Daily Scrum & Daily stand-up meetings for team synchronisation and progress tracking \\
        \hline
        Weekly Scrum & Weekly meetings for sprint planning and progress review \\
        \hline
        Jira & \makecell{Tracking progress, manage activities, and break activities into tasks.\\ Note: Not connected to GitHub Pull Requests} \\
        \hline
        Discord & Everyday communication and update announcements \\
        \hline
        GitHub & \makecell{Main repository for source code, approved project\\ documents, and the project’s requirements database} \\
        \hline
        SharePoint & \makecell{Documents not for public use go here, along with documents waiting\\ for review, and protocol templates.} \\
        \hline
        Visual Studio Code & Main environment for local development integrated with Git and GitHub \\
        \hline
        Modelio & Formal system modelling \\
        \hline
        draw.io & Flexible diagrams and illustrative modelling \\
        \hline
        gym-pybullet-drones & Simulating solution and testing scenarios \\
        \hline
        Overleaf (LaTeX) & Preparing plans and formal reports \\
        \hline
    \end{tabular}
    \caption{Tools and techniques.}
    \label{tab:tools}
\end{table}
\renewcommand{\arraystretch}{1.0}

\newpage
\subsection{Project Lifecycle \& Phases}
The activities carried out in this project are organised according to a combination of phases given in a course guide \cite{study-guide} (figure \ref{fig:phases-study-guide}) and a general project management model (figure \ref{fig:phases-general}) found in \cite{projektled}. The resulting project management model from this combination can be seen in figure \ref{fig:phases-modified}.

\begin{figure}[H]
    \centering
    \includegraphics[width=0.8\textwidth]{figures/phases-study-guide-V2.png}
    \caption{Project phases according to given course guide.}
    \label{fig:phases-study-guide}
\end{figure}

\begin{figure}[H]
    \centering
    \includegraphics[width=0.8\textwidth]{figures/phases-general-model-V2.png}
    \caption{Project phases according to general project management model.}
    \label{fig:phases-general}
\end{figure}

\begin{figure}[H]
    \centering
    \includegraphics[width=0.7\textwidth]{figures/phases-modified-model-V2.png}
    \caption{Project phases according to the project's modified project management model.}
    \label{fig:phases-modified}
\end{figure}

The project will follow a structured approach, with all key activities—quality, configuration, safety, requirements, and verification \& validation—managed through their respective plans: Quality Management Plan \cite{qm}, Configuration Management Plan \cite{cm}, Requirements Management Plan \cite{rm}, Safety Management Plan \cite{sm}, and V\&V Management Plan \cite{vvm}. Change management and quality assurance will follow the Configuration and Quality Management Plans. Given the critical role of requirements in the dependability domain, requirements specification will occur during the execution phase, with the Requirements Management Plan governing how they are captured, analysed, prioritised, and tracked. Safety, verification, and validation activities will be carried out according to the Safety and V\&V Management Plans. The project execution will follow a modified \gls{arp} V-model (figure \ref{fig:app-V-model}), ensuring systematic planning, oversight, and control throughout the lifecycle.
\begin{figure}[H]
    \centering
    \includegraphics[width=0.9\textwidth]{figures/V-model-full-modified-V2.png}
    \caption{V-model based on ARP4754A.}
    \label{fig:V-model}
\end{figure}

Larger versions of the above figures can be found under Figures in Appendix (figures \ref{fig:app-phases-study-guide}, \ref{fig:app-phases-general}, \ref{fig:app-phases-modified}, and \ref{fig:app-V-model}.