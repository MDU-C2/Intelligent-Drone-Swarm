\section{Related Documents}
\label{section:related-docs}
The following table is a collection of documents related to this project:
\newcolumntype{P}[1]{>{\raggedright\arraybackslash}p{#1}}
%\newcolumntype{A}[1]{>{\raggedright\let\newline\\\arraybackslash\hspace{0pt}}b{#1}}
%\newcolumntype{B}[1]{>{\centering\let\newline\\\arraybackslash\hspace{0pt}}m{#1}}
%\newcolumntype{C}[1]{>{\raggedright\let\newline\\\arraybackslash\hspace{0pt}}m{#1}}
\renewcommand{\arraystretch}{1.4}
\begin{table}[H]
    \centering
    \begin{tabular}{|B{1cm}|P{6.2cm}|P{4.5cm}|P{2.3cm}|}
        \hline
        \textbf{ID} & \textbf{Document Title} & \textbf{Owner (Role)} & \textbf{Current Version} \\
        \hline
        CM-01 & Configuration Management Plan \cite{cm} & Quality \& Configuration Manager & \\
        \hline
        QM-01 & Quality Management Plan \cite{qm} & Quality \& Configuration Manager & \\
        \hline
        RM-01 & Requirements Management Plan \cite{rm} & Requirements Manager & \\
        \hline
        SM-01 & Safety Management Plan \cite{sm} & Safety Manager & \\
        \hline
        VV-01 & Validation \& Verification Management Plan \cite{vvm} & Validation \& Verification Manager & \\
        \hline
    \end{tabular}
    \caption{Related documents within the project.}
    \label{tab:rel-docs}
\end{table}

\begin{table}[H]
    \centering
    \begin{tabular}{|P{4.5cm}|P{5cm}|P{6cm}|}
        \hline
        \textbf{ID} & \textbf{Title} & \textbf{Relevance} \\
        \hline
        IEEE Std 1012\texttrademark-2024 & IEEE Standard for System, Software, and Hardware Verification and Validation \cite{1012_2024} & \\
        \hline
        ISO/IEC/IEEE Std 29119-1:\texttrademark-2022 & Software and systems engineering - Software testing - Part 1: General Concepts \cite{29119_1_2022} & \\
        \hline
        ISO/IEC/IEEE Std 29119-4:\texttrademark-2021 &  Software and systems engineering - Software testing - Part 4: Test techniques \cite{29119_4_2021} & \\
        \hline
        ISO/IEC/IEEE Std 15288:\texttrademark-2023 & Systems and software engineering - System life cycle processes \cite{15288_2023} & \\
        \hline
        ISO/IEC/IEEE 29148:2018 & Systems and software engineering - Life cycle processes - Requirements engineering \cite{29148_2018} & \\
        \hline
        ISO 10007:2017 & Quality management - Guidelines for configuration management \cite{10007_2017} & \\
        \hline
        ISO 9001:2015 & Quality management systems — Requirements \cite{9001_2015} & \\
        \hline
        ISO/IEC 25002:2024 & Systems and software engineering — Systems and software Quality Requirements and Evaluation (SQuaRE) — Quality model overview and usage \cite{25002_2024} & \\
        \hline
        IEEE 730-2014 & IEEE Standard for Software Quality Assurance Processes \cite{730_2014} & \\
        \hline
        JAR-DEL-SRM-SORA-MB-2.5 & Specific Operations Risk Assessment (SORA) \cite{jarus_sora} & \\
        \hline
        ARP4761\texttrademark A & Guidelines for Conducting the Safety Assessment Process on Civil Aircraft, Systems, and Equipment \cite{sae_arp4761a} & \\
        \hline
    \end{tabular}
    \caption{Related standards and guidelines.}
    \label{tab:rel-standards}
\end{table}
\renewcommand{\arraystretch}{1.0}
\noindent For more information on related documents, see \cref{sec:deliverables} \nameref{sec:deliverables}.

\subsection{Management Plans}
Several specialised management plans support this project. Each addresses a critical aspect of the project lifecycle in more detail than this project plan can provide. Therefore, this project plan only references these areas at a high level, while dedicated management plans define the processes, responsibilities, and tools used.
\begin{itemize}
    \item \textbf{Configuration Management Plan} \cite{cm} – Defines the processes, tools, and responsibilities to manage project configurations and change management. Since the project involves multiple deliverables and contributions from several team members, strict configuration management is essential.
    \item \textbf{Quality Management Plan} \cite{qm} – Defines the standards, procedures, and responsibilities to maintain quality in processes, deliverables, and documentation. This is particularly important because the project is carried out in the dependability domain, where consistency and reliability are critical.
    \item \textbf{Requirements Management Plan} \cite{rm} – Because requirements specification is a major activity in the dependability domain, the team has decided that the requirement specification belongs to the execution phase rather than the planning phase. The Requirements Management Plan therefore governs how requirements are captured, analysed, prioritised, and tracked throughout the project.
    \item \textbf{Safety Management Plan} \cite{sm} – During the execution phase of the project, activities are organised according to \gls{sora} \cite{jarus_sora} and a modified version of \gls{arp}'s \cite{sae_arp4754a} V-model, where safety management is a central element. The Safety Management Plan specifies the procedures, responsibilities, and controls to identify, assess, and mitigate safety risks.
    \item \textbf{Validation \& Verification (V\&V) Management Plan} \cite{vvm} – In the modified \gls{arp} V-model adopted for this project, the V\&V process ensures both that the system is built correctly (verification) and that it meets the needs of the stakeholders (validation). It defines the approach, methods, and responsibilities for V\&V activities.
\end{itemize}
Together, these management plans provide the detailed guidance required to ensure that configuration, quality, requirements, safety, and validation \& verification are handled systematically throughout the project lifecycle.